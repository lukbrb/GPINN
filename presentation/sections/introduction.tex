\chapter{Introduction}\label{ch:intro}
In the realm of galactic dynamics, the overarching task of calculating the total potential of the galaxy could, in theory, be attained by summing the potentials of punctual masses corresponding to all the stars, given that the majority of the galaxy's mass resides in these stellar bodies. A typical galaxy boasts approximately $10^{11}$ stars, rendering the latter proposition largely unfeasible~\cite{binney2011galactic}. For the dynamic modeling of a galaxy, it generally suffices to accurately depict the galaxy's density field and to compute its gravitational potential via the Poisson equation. As is often the case in physics, analytical solutions—here, analytical potential-density pairs—are available for simplistic systems, yet in more realistic scenarios, numerical quadrature becomes a necessity~\cite{caravita_jeans_2021}. Numerical solutions frequently entail time-consuming calculations, especially when real galaxy dynamic modeling and Bayesian parameter estimation methods are in play~\cite{rigamonti2022maximally}.

Artificial Intelligence (AI) has ascended to become an indispensable part of modern scientific research across numerous domains. This ubiquity is largely justified by the enormous volume of data presently accessible. However, there exist fields wherein data quantity is limited. In such instances, one would aspire to leverage AI techniques to utilize the available data. An ingenious solution to this issue was proposed by~\cite{raissi_physics_2017}. The initial idea is to exploit the properties of neural networks (NNs) as universal function approximators, taking advantage of their auto-differentiation capability. Subsequently, the knowledge we possess in physics (symmetry, invariance, or conservation principles originating from the physical laws governing the observed data) is utilized to constrain the parameter space during the training phase of the neural network. This is referred to as Physics-Informed Neural Networks (PINNs). These PINNs can be utilized to solve nonlinear partial differential equations (PDEs) encountered in diverse physics problems—such as advection problems, diffusion problems, flow problems, etc. The PINNs can address a broad range of issues in computational sciences and leads to the introduction of new classes of numerical solvers for PDEs. Similar to grid-based simulations with adaptive mesh refinement, resolving the Poisson equation via the multigrid approach or Fourier methods is quite costly, the PINNs could potentially induce a paradigm shift in the study of complex systems by pushing hardware and computational boundaries.

The principal objective of this thesis is to enhance galactic modeling employing PINNs. We will study specific density profiles, commencing with simple analytical density profiles (see Section~\ref{sec:spherical-symmetry}), to then transition to more intricate, axisymmetric density profiles that are closer to reality (Section~\ref{sec:axisymmetry}). This endeavor could enable researchers working on galactic dynamics to perform parameter estimates with considerable time savings, opening previously untraversable paths. Other applications of PINNs in astrophysics exist, in a similar vein, they could be utilized to expedite the creation of initial conditions for galaxy simulations or complex system simulations.

We will lean on recent developments of PINNs to resolve the Poisson equation for the gravitational potential~\cite{kharazmi_variational_2019}. As a proof of concept, we will initially test the Hernquist density profile (see Section~\ref{sec:hernquist}). In this case, the gravitational potential can be calculated analytically and can be used to test and validate the accuracy of the PINNs. Once the ability of PINNs to resolve the Poisson equation is substantiated, we will attempt to resolve the potential for the Dehnen density profile (see Section~\ref{sec:dehnen}). While the Hernquist profile carries two parameters, mass and scale radius, the Dehnen profile has three: mass, scale radius, and the inner slope, denoted $\gamma$. The Dehnen profile has an analytical solution, but the Poisson equation now depends on an additional parameter, $\gamma$. With this second profile, we intend to assess the accuracy of PINNs in solving parametric differential equations.

Ultimately, we will proceed with more realistic, axisymmetric density profiles, such as the Thick Exponential Disk (Section~\ref{sec:disk}). Depending on the success of the latter, we could extend the project by applying the same technique to solve the Jeans equation of gravity, commonly used to predict galaxy velocity fields.

From the right choice of network architecture, to smart tricks for achieving convergence, or simplifying equations, the challenges faced in this project are numerous. The choice of architecture for the physics-informed neural networks is still an ongoing research topic, and could be the most demanding task. It's noteworthy to mention that to the best of our knowledge, no work has been done on solving the gravitational Poisson equation by PINNs.

This thesis is organized in six chapters. In Chapter~\ref{ch:galaxies-theory} is presented the theory behind galaxy dynamics, and particularly the Poisson equation applied to some models of interest. Chapter~\ref{ch:neural-networks} aims to give an overview of what neural networks are. Are also detailed some theorem or algorithms that justify the use of neural networks for approximating a gravitational potential. A definition of PINNs and their characteristics is given in Chapter~\ref{ch:pinns}. Finally, in Chapter~\ref{ch:applications} are presented the method with which we have used PINNs as well as the results obtained. We conclude in Chapter~\ref{ch:conclusion}. 

\par Also, as triviality is a vague concept that the best AI models could not define, the equations and calculations are detailed as much as possible for my fellow students who wish to go through this thesis. Some theorem or calculations can be found in the Appendices, and all the code used for this work is available on the following GitHub repository: \url{https://github.com/lukbrb/GPINN}  
