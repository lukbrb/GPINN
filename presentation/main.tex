\documentclass[12pt, aspectratio=169]{beamer}
\usepackage{hyperref}
\usepackage[T1]{fontenc}
\usepackage{graphicx} % Required for inserting images
\usepackage{amsmath, amsfonts}
\usepackage{apalike}
\usepackage{mathtools}
\usepackage[utf8]{inputenc}
\usepackage{subcaption}
\usepackage{tikz}
% --------------- Commandes ------------------
\newcommand{\dd}{\mathrm{d}}
\renewcommand{\L}{\mathcal{L}}
\renewcommand{\Vec}{\mathbf}
\newcommand{\moit}{0.5\textwidth}
% ---------------- Réglages ----------------
\hypersetup{
    colorlinks=false,
    linkcolor=blue,
    urlcolor=blue,
    citecolor=black,
    pdftitle={Application of Physics-Informed Neural Networks for Galaxy Dynamics} 
    }

\usetheme{Copenhagen}
\usecolortheme{beaver}
%\setbeamertemplate{navigation symbols}{} % cache boutons de navigation
\setbeamercovered{transparent}
\setbeamertemplate{headline}{} % cache sections en haut des diapos
\title{Application of Physics-Informed Neural Networks for Galaxy Dynamics}
\author{Lucas Barbier-Goy}
\date{June 5\textsuperscript{th} 2023}


% ---------------- Document ------------------
\begin{document}

\maketitle



\begin{frame}{Purpose}
    We aim to use PINNS for galaxy modelization. That is, after specifying all free parameters, the computation of:
    \begin{itemize}
        \item the surface brightness
        \item the line of sight velocity
        \item the line of sight velocity dispersion (the statistical dispersion of velocities about the mean velocity)
    \end{itemize}
    The solution of the Jeans equations (i.e. the velocity field) once projected along the line of sight can be directly compared to observational data. We take a first step by solving the Poisson equation for the gravitational potential $\Phi$ :
    $$ \nabla \Phi = 4 \pi G \rho $$
\end{frame}

\begin{frame}{Introduction}
    \tableofcontents
\end{frame}

%\begin{frame}{Intrdoduction}
%\begin{itemize}
%    \item Problématique
%    \item PINNs et NN
%    \item Applications
%    \item Résultats
%    \end{itemize}
%\end{frame}

%\begin{frame}{Exemple}
%    \begin{itemize}
%        \item <1-> Diapo 1
%        \item <2-> Diapo 2
%    \end{itemize}
%\only<2>{Seulement sur la seconde diapo}\only<1>{Seulement 1ere diapo}
%\end{frame}

%\section{Section 2}
%\begin{frame}{duo-colonne}
%    \begin{columns}
%        \column{0.5\textwidth} Colonne 1
%        \column{0.5 \textwidth} Colonne 2
%    \end{columns}
%\end{frame}

%\chapter{Introduction}\label{ch:intro}
In the realm of galactic dynamics, the overarching task of calculating the total potential of the galaxy could, in theory, be attained by summing the potentials of punctual masses corresponding to all the stars, given that the majority of the galaxy's mass resides in these stellar bodies. A typical galaxy boasts approximately $10^{11}$ stars, rendering the latter proposition largely unfeasible~\cite{binney2011galactic}. For the dynamic modeling of a galaxy, it generally suffices to accurately depict the galaxy's density field and to compute its gravitational potential via the Poisson equation. As is often the case in physics, analytical solutions—here, analytical potential-density pairs—are available for simplistic systems, yet in more realistic scenarios, numerical quadrature becomes a necessity~\cite{caravita_jeans_2021}. Numerical solutions frequently entail time-consuming calculations, especially when real galaxy dynamic modeling and Bayesian parameter estimation methods are in play~\cite{rigamonti2022maximally}.

Artificial Intelligence (AI) has ascended to become an indispensable part of modern scientific research across numerous domains. This ubiquity is largely justified by the enormous volume of data presently accessible. However, there exist fields wherein data quantity is limited. In such instances, one would aspire to leverage AI techniques to utilize the available data. An ingenious solution to this issue was proposed by~\cite{raissi_physics_2017}. The initial idea is to exploit the properties of neural networks (NNs) as universal function approximators, taking advantage of their auto-differentiation capability. Subsequently, the knowledge we possess in physics (symmetry, invariance, or conservation principles originating from the physical laws governing the observed data) is utilized to constrain the parameter space during the training phase of the neural network. This is referred to as Physics-Informed Neural Networks (PINNs). These PINNs can be utilized to solve nonlinear partial differential equations (PDEs) encountered in diverse physics problems—such as advection problems, diffusion problems, flow problems, etc. The PINNs can address a broad range of issues in computational sciences and leads to the introduction of new classes of numerical solvers for PDEs. Similar to grid-based simulations with adaptive mesh refinement, resolving the Poisson equation via the multigrid approach or Fourier methods is quite costly, the PINNs could potentially induce a paradigm shift in the study of complex systems by pushing hardware and computational boundaries.

The principal objective of this thesis is to enhance galactic modeling employing PINNs. We will study specific density profiles, commencing with simple analytical density profiles (see Section~\ref{sec:spherical-symmetry}), to then transition to more intricate, axisymmetric density profiles that are closer to reality (Section~\ref{sec:axisymmetry}). This endeavor could enable researchers working on galactic dynamics to perform parameter estimates with considerable time savings, opening previously untraversable paths. Other applications of PINNs in astrophysics exist, in a similar vein, they could be utilized to expedite the creation of initial conditions for galaxy simulations or complex system simulations.

We will lean on recent developments of PINNs to resolve the Poisson equation for the gravitational potential~\cite{kharazmi_variational_2019}. As a proof of concept, we will initially test the Hernquist density profile (see Section~\ref{sec:hernquist}). In this case, the gravitational potential can be calculated analytically and can be used to test and validate the accuracy of the PINNs. Once the ability of PINNs to resolve the Poisson equation is substantiated, we will attempt to resolve the potential for the Dehnen density profile (see Section~\ref{sec:dehnen}). While the Hernquist profile carries two parameters, mass and scale radius, the Dehnen profile has three: mass, scale radius, and the inner slope, denoted $\gamma$. The Dehnen profile has an analytical solution, but the Poisson equation now depends on an additional parameter, $\gamma$. With this second profile, we intend to assess the accuracy of PINNs in solving parametric differential equations.

Ultimately, we will proceed with more realistic, axisymmetric density profiles, such as the Thick Exponential Disk (Section~\ref{sec:disk}). Depending on the success of the latter, we could extend the project by applying the same technique to solve the Jeans equation of gravity, commonly used to predict galaxy velocity fields.

From the right choice of network architecture, to smart tricks for achieving convergence, or simplifying equations, the challenges faced in this project are numerous. The choice of architecture for the physics-informed neural networks is still an ongoing research topic, and could be the most demanding task. It's noteworthy to mention that to the best of our knowledge, no work has been done on solving the gravitational Poisson equation by PINNs.

This thesis is organized in six chapters. In Chapter~\ref{ch:galaxies-theory} is presented the theory behind galaxy dynamics, and particularly the Poisson equation applied to some models of interest. Chapter~\ref{ch:neural-networks} aims to give an overview of what neural networks are. Are also detailed some theorem or algorithms that justify the use of neural networks for approximating a gravitational potential. A definition of PINNs and their characteristics is given in Chapter~\ref{ch:pinns}. Finally, in Chapter~\ref{ch:applications} are presented the method with which we have used PINNs as well as the results obtained. We conclude in Chapter~\ref{ch:conclusion}. 

\par Also, as triviality is a vague concept that the best AI models could not define, the equations and calculations are detailed as much as possible for my fellow students who wish to go through this thesis. Some theorem or calculations can be found in the Appendices, and all the code used for this work is available on the following GitHub repository: \url{https://github.com/lukbrb/GPINN}  

%\chapter{Basics of Galactic Dynamics}\label{ch:galaxies-theory}
Galactic dynamics is a branch of astrophysics that studies the distribution of masses and their movements in galaxies. Galaxies are primarily composed of stars, gas, dust, and dark matter. Modeling the dynamics of galaxies allows for a better understanding of the evolution, formation, and interactions between galaxies. In this chapter, we introduce some basic concepts about gravitational fields, and in particular, the Poisson equation, which will guide the solution to the problems we will address in Chapter~\ref{ch:applications}. Finally, we  discuss spherical symmetry profiles and axisymmetric profiles.

\section{Gravitational Potential}\label{sec:gravity}
Gravity is the dominant force governing movements on the scale of galaxies. Particularly, most of a galaxy's mass is found in the stars composing it. To determine the total gravitational field of the galaxy, one could imagine summing the contribution of each of these stars.
\par However, this method is not feasible in practice due to the large number of stars\footnote{A typical galaxy contains about $10^{11}$ stars.} in a typical galaxy. For most cases, it is sufficient to smooth the mass density of the stars on a scale small compared to the size of the galaxy, but large compared to interstellar distance~\cite{binney2011galactic}.

\subsection{General Results}

To determine how we will compute the gravitational field given the mass density, we first look at the basic theory. Initially, the goal is to compute the force $\Vec{F}(x)$ exerted by the gravitational attraction generated by a mass distribution $\rho(\Vec{x'})$ on a particle of mass $m_s$ at position $\Vec{x}$. This force is obtained by summing all the small contributions $\delta\Vec{F}(x)$, defined as
\begin{equation}
\label{eq:fg_small_contrib}
\delta\Vec{F}(x) = Gm_s \cdot \frac{\Vec{x'} - \Vec{x}}{|\Vec{x'} - \Vec{x}|^3}\delta m(\Vec{x'}) = Gm_s \cdot \frac{\Vec{x'} - \Vec{x}}{|\Vec{x'} - \Vec{x}|^3}\rho(\Vec{x'}) \dd^3 \Vec{x'}\text{ ,}
\end{equation}
to the global force generated by each volume element $\dd^3 x'$ located at $\Vec{x'}$. This is written as follows:
\begin{equation}
\label{eq:fg_def}
\Vec{F}(x) = m\Vec{g}(x) \text{ where } \Vec{g}(x) \equiv G \int \dd^3 x' \frac{\Vec{x'} - \Vec{x}}{|\Vec{x'} - \Vec{x}|^3}\rho(\Vec{x'})\text{,}
\end{equation}
where $G$ is naturally the gravitational constant.
The gravitational field $\Vec{g}(x)$ is related to the gravitational potential $\Phi(x)$ by the following relation:
\begin{equation}
\label{eq:gfield_to_phi}
\Vec{g}(x) = - \Vec{\nabla}_x \Phi(x)
\end{equation}

The potential is convenient because it is a scalar field, easier to visualize and manipulate than the gravitational field, but containing the same information.

\subsection{Poisson's Equation}

By taking the divergence of the gravitational field $\Vec{g}$, we can also show that all contributions must come from the point $x=x'$. Therefore, we can restrict the integration volume to a small sphere of radius $h$ centered at point $x$. Since $h$ is small, we can consider $\rho(\Vec{x})$ constant across the volume of the sphere. This allows us to extract it from the following integral~\cite{binney2011galactic}:

\begin{equation}
    \begin{split}
    \label{eq:demo_eq_poisson}
    \Vec{\nabla}\cdot \Vec{g}(x) &= G\rho (x) \int_{r \leq h} \dd^3 x' \Vec{\nabla}_x \cdot \left(\frac{x'-x}{|x'-x|^3}\right)\\
                                 &= - G\rho (x) \int_{r \leq h} \dd^3 x' \Vec{\nabla}_x' \cdot \left(\frac{x'-x}{|x'-x|^3}\right)\\
                                 &= - G\rho (x) \int_{r = h} \dd^2 \Vec{S}'\cdot \left(\frac{x'-x}{|x'-x|^3}\right)\\
    \end{split}
\end{equation}
where we have set $r=|x'-x|$ to lighten the notation. Since at the surface of a sphere we have $\dd^2 \Vec{S} = h (x'-x)  \dd^2 \Omega$, Equation~\eqref{eq:demo_eq_poisson} becomes:
\begin{equation}
    \label{eq:demo_poisson2}
    \Vec{\nabla}\cdot \Vec{g}(x) = - G\rho (x) \int \dd^2 \Omega = -4\pi G\rho(\Vec{x})
\end{equation}

Finally, by substituting Equation~\eqref{eq:gfield_to_phi} into Equation~\eqref{eq:demo_poisson2}, we arrive at Poisson's equation, which links the gravitational potential to the mass distribution density $\rho(\Vec{x})$:

\begin{equation}
\label{eq:poisson_eq_general}
\nabla^2 \Phi(\Vec{x}) = 4 \pi G \rho(\Vec{x})
\end{equation}

\par Poisson's equation is a differential equation that provides $\Phi(\Vec{x})$ given $\rho(\Vec{x})$ and appropriate boundary conditions. In our context, we will consider isolated systems for which $\Phi \rightarrow 0$ as $|\Vec{x}| \rightarrow \infty$.

\section{Spherically Symmetric Profiles}\label{sec:spherical-symmetry}
Spherically symmetric profiles are a simplified approximation suitable for certain galaxies. In such cases, the mass distribution and the potential depend only on the radial distance $r$ from the galaxy center, and the Laplacian in Poisson's equation simplifies to:

\begin{equation}
\label{eq:poisson_spherique}
\nabla^2 \Phi(r) = \frac{1}{r^2}\frac{\dd}{\dd r}\left(r^2\frac{\dd \Phi}{\dd r}\right)
\end{equation}

Solving Poisson's equation for spherical profiles depends on the specific form of the mass distribution density $\rho(r)$. We will delve into the distributions of Hernquist~\cite{hernquist_analytical_1990} and Dehnen~\cite{dehnen_family_1993} profiles.

\subsection{Hernquist Profile}
The observed luminosity distribution of numerous galactic bulges and elliptical galaxies is well characterized by the empirical law
\begin{equation}
\label{eq:deVaucouleur}
\log_{10} \left[ \frac{I(R)}{I(R_e)}\right] = -3.331 \left[ \left(\frac{R}{R_e}\right)^{1/4} - 1\right]\text{,}
\end{equation}
where $R$ is the radius projected onto the plane of the sky, $R_e$ is the effective radius of the isophote enclosing half the light, and $I$ is the surface brightness~\cite{deVaucouleurs1948}. The Hernquist potential-density pair~\cite{hernquist_analytical_1990} retains the same properties as Equation~\eqref{eq:deVaucouleur}, but allows for analytical solutions. The density proposed by Hernquist is:

\begin{equation}
\label{eq:hernquist_density}
\rho(r) = \dfrac{M}{2\pi}\dfrac{a}{r}\dfrac{1}{(r+a)^3}\text{,}
\end{equation}
where $M$ is the total mass of the system and $a$ is a scalelength. For such a density, Poisson's Equation~\eqref{eq:poisson_spherique} takes the following form:

\begin{equation*}
\dfrac{1}{r^2} \dfrac{\partial}{\partial r}\left(r^2 \dfrac{\partial \Phi}{\partial r}\right) = 4\pi G \left[\dfrac{M}{2\pi}\dfrac{a}{r}\dfrac{1}{(r+a)^3}\right]
\end{equation*}
Given that the potential-density pair depends only on the radial coordinate $r$, the previous equation can be rewritten as
\begin{equation*}
\dfrac{1}{r} \dfrac{\dd}{\dd r}\left(r^2 \dfrac{\dd \Phi}{\dd r}\right) = \dfrac{2GMa}{a^3(\frac{r}{a}+1)^3}
\end{equation*}

To later facilitate the computation of this potential, we aim to make this equation dimensionless. We thus set $s = \frac{r}{a}$ and $\Phi'(r) = \Phi(r)/\frac{GM}{a}$. We then get the dimensionless Poisson equation in the case of Hernquist's density:

\begin{equation}
\label{eq:residual_poisson_hernquist1}
\boxed{\dfrac{\dd}{\dd s}\left(s^2 \dfrac{\dd \Phi'}{\dd s}\right) = \dfrac{2s}{(s+1)^3}}
\end{equation}

The solution to this equation can be calculated analytically (see Appendix~\ref{app:poisson}), resulting in:

\begin{equation}
\label{eq:hernquist_pot}
\Phi(s) = - \frac{1}{s + 1}
\end{equation}

\subsection{Dehnen Profile}
Dehnen's profiles~\cite{dehnen_family_1993} are a family of potential-density pairs describing spherical galaxies and bulges. The model includes an additional parameter, $\gamma \in [0, 3[$ representing the inner slope of the model. Hernquist's~\cite{hernquist_analytical_1990} and Jaffe's~\cite{jaffe1983simple} models are included in Dehnen's family of potential-density pairs as special cases when $\gamma=1$ and $\gamma=2$, respectively. Finally, in this case, the density can be written as follows:

\begin{equation}
    \label{eq:density-dehnen}
    \rho(r) = \dfrac{(3-\gamma)M}{4\pi}\dfrac{a}{r^{\gamma}(r+a)^{4-\gamma}}\text{,}
\end{equation}

Performing the same procedure as for the Hernquist model, it can be easily shown that the associated dimensionless Poisson equation can be written as:

\begin{equation}
\label{eq:poisson-dehnen}
\dfrac{\dd}{\dd s}\left(s^2 \dfrac{\dd \Phi'}{\dd s}\right) = \dfrac{2s^{2-\gamma}}{(1+s)^{4-\gamma}}
\end{equation}
which has the following solution~\cite{dehnen_family_1993}:

\begin{equation}
    \label{eq:pot-dehnen}
    \Phi(s) = 
    \begin{cases} 
    -\dfrac{1}{2 - \gamma} \left[1 - \left( \dfrac{s}{1 + s} \right )^{2-\gamma}\right] & \text{if } \gamma \neq 2 \\
    \\
    \ln \left(\dfrac{s}{1 + s}\right) & \text{if } \gamma = 2
    \end{cases}
\end{equation}

\section{Axisymmetric Profiles}\label{sec:axisymmetry}

Axisymmetric profiles offer a better approximation for galaxies that exhibit more complex shapes, such as spiral galaxies. In this case, the mass distribution depends on both the radius $R$ in the equatorial plane and the vertical coordinate $z$ perpendicular to this plane. The Laplacian in the Poisson equation then takes the following form:

\begin{equation}
\label{eq:poisson_axisymmetrique}
\nabla^2 \Phi(R, z) = \frac{1}{R}\frac{\partial}{\partial R}\left(R\frac{ \partial \Phi}{\partial R}\right) + \frac{\partial^2\Phi}{\partial z^2}
\end{equation}

Certain axisymmetric profiles admit analytical solutions, such as the Kuzmin profile or the Miyamoto-Nagai profile~\cite{miyamoto1975three}. More generally, solving the Poisson equation for this class of profiles requires numerical methods. In particular, we will be interested in Section~\ref{sec:disk} in the model of the thick exponential disk, which does not admit an analytical solution.

\subsection{Exponential Disk}

The mass distribution of the stellar disk of most galaxies is well represented by an exponential radial profile~\cite{freeman1970disks}

\begin{equation*}
\Sigma(R) = \Sigma_0 \exp{\left(-\frac{R}{R_d}\right)}\text{,}
\end{equation*}
where $\Sigma$ is the surface density, $\Sigma_0$ the surface density at the center of the disk, $R$ is the radius within the disk, and $R_d$ is a scalelength of the disk. These disks can have variable vertical distributions, which are commonly modeled with a hyperbolic secant function $\text{sech}^n = \cosh^{-n}$. Thus, to fully describe galaxy disks, the following formula is used~\cite{smith2015simple}:

\begin{equation}
\label{eq:exp_disk_general}
\rho(R, z) = \rho_0 \exp{\left(-R/R_d\right)}\cdot \text{sech}^n\left(-|z|/z_d\right)\text{,}
\end{equation}
where $z_d$ is a  scaleheight and $n$ is typically between  $\sim 1$ and $3$. $\rho_0$ is the central mass density of the disk, and is usually normalized by the scalelengths of the model. We note the special case $n \rightarrow \infty$ which describes the doubly exponential disk; the vertical component also decreases exponentially.

\subsection{Thick Exponential Disk}

In our study, we are interested in the case of the thick exponential disk. Particularly we study the particular case when $n=2$. Equation~\eqref{eq:exp_disk_general} thus becomes:

\begin{equation}
    \label{eq:exp-disk}
    \rho(R, z) = \rho_0 \exp{\left(-R/R_d\right)}\cdot \cosh^{-2}{\left(-|z|/z_d\right)}
\end{equation}

Firstly, we want to find the associated dimensionless Poisson equation, and then we will look at the approximate solution of this equation~\eqref{eq:exp-disk-potential}. Using the Laplacian from equation~\eqref{eq:poisson_axisymmetrique}, we get directly:

\begin{equation}
\label{eq:exp-disc-poisson1}
\dfrac{1}{R} \dfrac{\partial}{\partial R} \left( R \dfrac{\partial \Phi}{\partial R}\right) + \dfrac{\partial^2 \Phi}{\partial z^2} = 4\pi G \rho_0 \exp{\left(-R/R_d\right)}\cdot \cosh^{-2}{\left(-|z|/z_d\right)}
\end{equation}
Now, by setting $z' = z/z_d$ and $R' = R/R_d$, we can rewrite~\eqref{eq:exp-disc-poisson1} in the form:

\begin{equation*}
\dfrac{1}{R_{d}^{2}} \dfrac{1}{R'} \dfrac{\partial}{\partial R'} \left(R' \dfrac{\partial \Phi}{\partial R'}\right) + \dfrac{1}{z_{d}^{2}}\dfrac{\partial^2 \Phi}{\partial z'^2} = 4\pi G \rho_0 \exp{-R'} \cosh^{-2}{z'}
\end{equation*}
Finally, if we set $\eta = z_d/R_d $ and $\phi'= \frac{\phi}{G M_d/z_d}$, we get a dimensionless Poisson equation for the thick exponential disk:

\begin{equation}
\label{eq:poisson-exp-disc-final}
\dfrac{1}{R'} \dfrac{\partial}{\partial R'} \left(R' \dfrac{\partial \Phi'}{\partial R'}\right) + \dfrac{1}{\eta^{2}}\dfrac{\partial^2 \Phi'}{\partial z'^2} = e^{-R'} \cosh^{-2}{z'}
\end{equation}

In order to implement a numerical solution for this equation, we follow the results of Appendix A of~\cite{bonetti2021dynamical}. The potential is given by the following equation:

\begin{equation}
\label{eq:exp-disk-potential}
\Phi(R, z) = - 2\pi G \alpha \rho_0 \int_{0}^{\infty} dk J_0 (kR) \dfrac{I_z (k)}{(\alpha^2 + k^2)^{\frac{3}{2}}}\text{,}
\end{equation}
with
\begin{equation*}
I_z(k) = \dfrac{4}{\beta} \left\{ 1 - \dfrac{k}{k+\beta} \left[ e^{-z\beta} {}_2F_1\left(1, 1 + \frac{k}{\beta}; 2 + \frac{k}{\beta}; -e^{-z\beta} \right) + e^{z\beta} {}_2F_1\left(1, 1 + \frac{k}{\beta}; 2 + \frac{k}{\beta}; -e^{z\beta} \right)\right] \right\}
\end{equation*}
where $_2F_1$ is the Gaussian hypergeometric function, defined such that :

\begin{equation}
    \label{eq:hypergeometric}
    _2F_1(a, b, c; z) = \sum_{n=0}^{\infty} \frac{(a)_n (b)_n}{(c)_n}\cdot \dfrac{z^n}{n!}
\end{equation}
where $(a)_n$ is the Pochhammer symbol and $|z| < 1$. Note that we also set $\alpha=1/R_d$ and $\beta = 2/z_d$.



\section{Neural Networks}\label{sec:neural-networks}

\begin{frame}{Perceptron}
    \begin{columns}
        \column{\moit} \textbf{Perceptron}: Model inspired by the functioning of neurons in the human brain, provide a mathematical model that enabled machines to learn from data and make predictions~\cite{rosenblatt1958perceptron}.
        \column{\moit} \begin{figure}[ht]
                                \centering
                                \includegraphics[width=\textwidth]{imgs/Single-Perceptron.png}
                                \caption{Illustration of an LTU, which are the building blocks of neural networks}
                                \label{fig:perceptron}
                            \end{figure}
    \end{columns}
\end{frame}


\begin{frame}{Neural networks}
    \begin{columns}
        \column{\moit} Also known as \emph{multi-layer} perceptron. 
                    \begin{enumerate}
                        \item Input layer for the data $\Vec{x}$
                        \item Hidden layers
                        \item Output layer for the network's prediction $\hat{f}_{\Vec{x}}$
                    \end{enumerate}{}
                     Output of layer $l$ can be expressed as:
                        \begin{equation*}
                            \label{eq:output-any-layer}
                            \Vec{a}^{(l)} = \sigma^{(l)}\left(\mathbf{W}^{(l)} \cdot \Vec{a}^{(l-1)} + \Vec{b}^{(l)}\right)
                        \end{equation*}
        \column{\moit} \begin{figure}[ht]
                            \centering
                            \includegraphics[width=\textwidth]{imgs/Architecture-perceptron-multi-couches-2.png}
                            \caption{Structure of a neural network with three hidden layers.}
                            \label{fig:mlp}
                        \end{figure}
                       
    \end{columns}
\end{frame}

\begin{frame}{Neural Networks}
    A neural network is a mathematical function. Can be described as a series of nested non-linear functions:

\begin{align}
f &\text{: } \mathbb{R}^n \to \mathbb{R}^m\nonumber\\
f &= g \circ f_L \circ f_{L-1} \dots f_2 \circ f_1 (x) \text{ with, }
\end{align}

\begin{align}
f_l &\text{: } \mathbb{R}^{n_l} \to \mathbb{R}^{n_{l-1}}\nonumber\\
f_l&(x) = \sigma^{(l)}(\mathbf{W}^{(l)} \cdot \Vec{x} + b^{(l)})
\end{align}

$n$ : dimension of the input $\Vec{x}$, $m$ : dimension of the output $\Vec{y}$, $g$: \emph{output function}
\end{frame}


\begin{frame}{Loss Function}
    Quantification of the error produced by the network via a \emph{loss function}. Commonly used Mean Squared Error (MSE):
    \begin{equation*}
        \label{eq:def-loss-function}
        \mathcal{L}(\theta) = \frac{1}{2n} \sum_{i=1}^n \left[\hat{y}_{\theta}^{(i)} - y^{(i)}\right]^2
    \end{equation*}
    $\theta$: set of parameters (weights and biases), $\hat{y} = \hat{f}(\Vec{x})$: value predicted, $y=f(\Vec{x})$: \emph{real} value.
\end{frame}


\begin{frame}{Gradient Descent}
    Method to udpdate network's parameters: \emph{Gradient descent}~\cite{cauchymethode}
    \begin{columns}
        \column{\moit} Parameters $\theta$  at step $n+1$ are updated:
            \begin{equation*}
                    \label{eq:gradient-descent}
                    \Vec{\theta}_{n+1} = \Vec{\theta}_n - \eta \nabla \mathcal{L}(\Vec{\theta}_n)
            \end{equation*}
            $\eta \in \mathbb{R}^+$ is called \emph{learning rate}.
        \column{\moit} 
        \begin{figure}
            \centering
            \includegraphics[width=\textwidth]{imgs/desc-grad.png}
            \caption{Influence of the learning rate on the convergence of the loss.}
        \end{figure}
    \end{columns}
\end{frame}


%\begin{frame}{Back-propagation}



%\only<1>{Algorithm that enables the computation of the gradient of the error, thereby permitting the adjustment of network parameters according to~\eqref{eq:gradient-descent}.
%\begin{figure}
%    \centering
%    \begin{tikzpicture}

%        \node[draw,rectangle,minimum height=2cm,minimum width=3cm,align=center] (layer) at (0,0) {Layer $l$};
        
%         \draw[<-] ([yshift=0.5cm]layer.west) -- ++(-3,0) node[midway,above] {$X^{(l)}=a^{(l-1)}$};
%         \draw[->] ([yshift=-0.5cm]layer.west) -- ++(-3,0) node[midway,below] {$\frac{\partial E}{\partial X^{(l)}}$};
        
%         \draw[->] ([yshift=0.5cm]layer.east) -- ++(3,0) node[midway,above] {$a^{(l)}=X^{(l+1)}$};
%         \draw[<-] ([yshift=-0.5cm]layer.east) -- ++(3,0) node[midway,below] {$\frac{\partial E}{\partial a^{(l)}} $};
%     \end{tikzpicture}
%     \caption{Hidden layer $l$ of a neural network.}
%     \label{fig:illustration-layer-l}
% \end{figure}}

% \only<2>{The  algorithm can be summarized in three main steps:
% \begin{enumerate}
%     \item Compute the forward pass for each input-output pair by proceeding from layer 1, the input layer, to layer $L$, the output layer.
%     \item Compute the backpward phase for each input-output pair by proceeding from layer $L$, the output layer, to layer 1, the input layer. 
%     %\begin{enumerate}
%     %   \item Evaluate the error term for the final layer.
%     %   \item Backpropagate the error terms for the hidden layers, starting from the last hidden layer $l=L-1$.
%     %   \item Evaluate the partial derivatives of the error with respect to $w_{ik}^{(l)}$ 
%     %\end{enumerate}
%     \item Update the parameters according to the gadient descent algorithm.
% \end{enumerate}

% One full cycle of this algorithm is called an \emph{epoch}.}
% \end{frame}

\begin{frame}{Important Properties}

    \begin{itemize}
        \item \textbf{Universal Approximation Theorem}: asserts that a neural network with a single hidden layer can approximate any continuous function on compact subsets of $\mathbb{R}^n$, provided that the hidden layer's activation function is non-constant, bounded, and continuous.
        \item \textbf{Automatic Differentiation}: Efficient way of computing derivatives with no approximation error.
    \end{itemize}
    
\end{frame}

\section{Physics-Informed Neural Networks}\label{sec:pinns}


\begin{frame}{Physics-Informed Neural Network (PINN)}
\begin{itemize}
    \item Specificity of PINNs: the loss function takes the form of a physical equation
    \item The PINN method is a meshless solution technique, finding the solutions by minimizing a loss function
    \item Ability to solve problems with very little, or noisy data.
    
\end{itemize}
    
\end{frame}



\begin{frame}{Definition of a PINN}
    \begin{enumerate}
        \item Solution $u(z)$ is approximated by a neural network parameterized by a set of parameters $\theta$
        \item For a general differential equation 
        \begin{equation*}
            \label{eq:pde-exemple}
            u_t + \mathcal{F}[u;\lambda] = 0\text{, } x\in \Omega\text{, } t \in [0, T]\text{,}
        \end{equation*} we set
        \begin{equation*}
            p \coloneqq u_t - \mathcal{F}[u;\lambda] \text{.}
        \end{equation*} This function $p$ is referred to as a \emph{physics-informed neural network}.
        \item The network learns to approximate the solution by finding the set of parameters $\theta$ that minimizes a loss function $\mathcal{L}(\theta)$
    \end{enumerate}
\end{frame}

\begin{frame}{Specific Loss Function}
    In the case of a PINN, the loss function is a sum of three components:

    \begin{itemize}
        \item The PDE residual loss 
        \item The boundary loss
        \item The data loss
    \end{itemize}

    \begin{equation*}
        \label{eq:loss-galaxy}
        \mathcal{L}(\theta) = \dfrac{1}{N_c}\sum^{N_c}_{i=1} \left\|\nabla^2 \hat{\Phi}(z_i) - 4 \pi G \rho(z_i) \right\|^2 + \dfrac{1}{N_{d}}\sum^{N_{d}}_{i=1} \left\|\hat{\Phi}(z_i) - \Phi_i \right\|^2
    \end{equation*}


\end{frame}

% \begin{frame}{Applications}
%     \begin{itemize}
%         \item Forward Problem
%         \item Inverse Problem
%     \end{itemize}
% \end{frame}
\begin{frame}{Summary}
    \begin{figure}[ht]
    \centering
    \includegraphics[scale=0.2]{imgs/training-pinn-schema.png}
    \caption{Structure and training procedure of a PINN. \textit{Illustration from~\cite{cuomo_scientific_2022}}.}
    \label{fig:loss-pinn}
\end{figure}

\end{frame}

\section{Applications}\label{ch:applications}


\subsection{Hernquist Model}\label{sec:hernquist}

\begin{frame}{Hernquist Model}
        The differential equation  we aim to solve:

    \begin{equation*}
        \dfrac{\dd}{\dd s}\left(s^2 \dfrac{\dd \Phi'}{\dd s}\right) = \dfrac{2s}{(s+1)^3}
    \end{equation*} which yields the following loss function for the PINN:
    \begin{equation*}
            \mathcal{L}(\theta) = \dfrac{1}{N_c}\sum^{N_c}_{i=1} \left|\dfrac{\dd}{\dd s_i}\left(s_{i}^{2} \dfrac{\dd \hat{\Phi}'}{\dd s_i}\right) - \dfrac{2s_i}{(s_i+1)^3} \right|^2 + \dfrac{1}{N_d}\sum^{N_d}_{i=1} \left|\hat{\Phi}'(s_i) - \Phi'_i \right|^2
    \end{equation*}
\end{frame}

\begin{frame}{Hernquist - Training}
\begin{columns}
    \column{\moit}
    \begin{figure}
        \centering
        \includegraphics[width=\textwidth]{imgs/training-points-hernquist.png}
        \caption{Configuration of training points in the spatial domain $s \in [0, 1000]$.}
        \label{fig:training-points-hernquist}
    \end{figure}
    \column{\moit}
    \begin{figure}
        \centering
        \includegraphics[width=\textwidth]{imgs/error-hernquist.png}
        \caption{Evolution of training and validation error functions.}
        \label{fig:losses-hernquist}
    \end{figure}
\end{columns}
\end{frame}

\begin{frame}{Hernquist - Results}

\begin{figure}
    \centering
        \begin{subfigure}[b]{0.49\textwidth}
        \centering
        \includegraphics[width=\textwidth]{imgs/test-plot-hernquist.png}
        \caption{Comparison between the actual value of the potential and that predicted by the PINN on a test domain $s \in [0, 100]$.}
        \label{fig:test-plot-hernquist}
        \end{subfigure}
    \hfill
    \begin{subfigure}[b]{0.49\textwidth}
        \centering
        \includegraphics[width=\textwidth]{imgs/relative-error-hernquist.png}
        \caption{Relative error along the domain $s$. The average error over the entire domain is $1.71$\%.}
        \label{fig:relative-error-hernquist}
    \end{subfigure}
    \caption{As illustrated by the two figures presented, a PINN is capable of predicting with acceptable accuracy the value of the Hernquist potential $\Phi$ at a given point in a domain in which it has been trained.}
    \label{fig:three graphs}
\end{figure}
\end{frame}


\begin{frame}{Dehnen Model}
\only<1>{The differential equation we want to solve:

\begin{equation*}
    \dfrac{\dd}{\dd s}\left(s^2 \dfrac{\dd \Phi'}{\dd s}\right) = \dfrac{2s^{2-\gamma}}{(1+s)^{4-\gamma}}
\end{equation*}}

The exponent $\gamma$ drastically changes the dynamics of the potential:
\only<2>{
\begin{figure}
\centering
\includegraphics[width=0.6\textwidth]{imgs/gamma-vs-x0-dehnen.png}
\caption{Value of the gravitational potential at $s_0 = 0.01$ for different values of $\gamma \in [0, 3[$. Predicting the potential value at $s_0=0$ is a complex task for the PINN given the high sensitivity of $\Phi(s_0, \gamma)$ to the value of $\gamma$.}
\label{fig:gamma-vs-x0-dehnen}
\end{figure}}
\end{frame}
\subsection{Dehnen Model}\label{sec:dehnen}


\begin{frame}{Dehnen - Training}
    \begin{columns}
        \column{\moit} \begin{figure}
            \centering
            \includegraphics[width=\textwidth]{imgs/training-points-dehnen.png}
            \caption{Distribution of training and validation points on the domain.}
            \label{fig:training-points-dehnen}
        \end{figure}
        \column{\moit} \begin{itemize}
            \item Problems to obtain satisfactory results
            \item No observable improvement by manually tuning hyperparameters
            \item Change the penality of loss function to MAE
        \end{itemize}
    \end{columns}
\end{frame}



\begin{frame}{Dehnen - Results}
    \begin{columns}
    \column{\moit}
        \begin{figure}
            \centering
            \includegraphics[width=\textwidth]{imgs/test-plot-dehnen.png}
            \caption{Comparison between the real value of the potential and that predicted by the PINN on a test domain $s \in [0, 10]$ for different values of $\gamma$.}
            \label{fig:test-plot-dehnen}
        \end{figure}
    \column{\moit} 
        \begin{figure}
            \centering
            \includegraphics[width=\textwidth]{imgs/relative-error-dehnen.png}
            \caption{Relative error on the $s \times \gamma$ grid. The relative error can reach nearly 20\%.}
            \label{fig:relative-error-dehnen}
        \end{figure}
    \end{columns}
\end{frame}



\subsection{Thick Exponential Disk}\label{sec:disk}

\begin{frame}{Thick Exponential Disk Model}
    We aim to solve the following differential equation:
    \begin{equation*}
        \dfrac{1}{R'} \dfrac{\partial}{\partial R'} \left(R' \dfrac{\partial \Phi'}{\partial R'}\right) + \dfrac{1}{\eta^{2}}\dfrac{\partial^2 \Phi'}{\partial z'^2} = e^{-R'} \cosh^{-2}{z'}
\end{equation*} 
$z' = z/z_d$, $R' = R/R_d$, $\eta = z_d/R_d $ and $\phi'= \frac{\phi}{G M_d/z_d}$
\end{frame}

\begin{frame}{Thick Exponential Disk - Training}
    \begin{columns}
        \column{0.4\textwidth} 
        \begin{itemize}
            \item Wish to obtain the best solution possible
            \item Need to find the best set of hyperparameters
            \item Wish to understand the impact of certain hyperpameters on the error

        \end{itemize}
        \column{0.6\textwidth}
        Simple grid search for hyperparameters fine-tuning
        \begin{table}[h]
        \centering
        \scalebox{0.8}{
        \begin{tabular}{|l|c|}
        \hline
        \textbf{Parameters} & \textbf{Values} \\
            \hline
            \# Neurons & 32, 64, 128 \\
            \hline
            \# Layers & 1, 2, 3, 4, 5, 6 \\
            \hline
            Learning Rate & 1e-4, 1e-5, 1e-6 \\
            \hline
            Loss Func. & mse \\
            \hline
            Activation & Tanh, Sigmoid, SiLU, LogSigmoid \\
            \hline
        \end{tabular}}
        \label{tab:fine-tuning}
        \end{table}
    \end{columns}
\end{frame}

\begin{frame}{Thick Exponential Disk - Results}
    \only<1>{
    Two kind of solutions:
    \begin{enumerate}
        \item Computing an approximate solution efficiently $\rightarrow 3 \times 32$ NN, ~ 80 seconds of computation and average error of $1.85\%$.
        \item Find a model giving the smallest error possible $\rightarrow$ fine-tuning, $6 \times 128$ NN, ~ 20 minutes of training and average error of $0.36\%$.
    \end{enumerate}}
    \only<2>{
    \begin{figure}
        \centering
        \includegraphics[width=0.6\textwidth]{imgs/relative-error-expdisc.png}
        %\caption{Relative error on the $R' \times z'$ grid. The average error over the entire domain is 0.36\%, and the maximum error is 0.99\%.}
        \label{fig:relative-error-expdisc}
    \end{figure}
    Satisfactory result, but parameter $\eta$ is here fixed $\rightarrow$ Need to extend the PINN !
    }
    
\end{frame}

\begin{frame}{Hyperparameters}
    How do hyper-parameters influence the relative error? 
    \only<1>{The activation function:
    \begin{figure}
        \centering
        \includegraphics[scale=0.5]{imgs/function-error-expdisc.png}
        %\caption{Average relative error for certain activation functions.}
        \label{fig:function-error-expdisc}
    \end{figure}
}
    \only<2>{The learning rate:
    \begin{figure}
        \centering
        \includegraphics[scale=0.5]{imgs/learning-rate-expdisc.png}
        %\caption{Average relative error for different learning rates.}
        \label{fig:learning-rate-expdisc}
    \end{figure}
    }

    \only<3>{The number of neurons per layer:
     \begin{figure}
        \centering
        \includegraphics[scale=0.4]{imgs/neurons-error-expdisc.png}
        %\caption{Influence of the number of neurons per hidden layer on the average relative error.}
        %\label{fig:neurons-error-expdisc}
    \end{figure}
    }
    \only<4>{The number of hidden layers:
    \begin{figure}
        \centering
        \includegraphics[scale=0.4]{imgs/layers-error-expdisc.png}
        %\caption{Influence of the number of hidden layers on the average relative error.}
        %\label{fig:layers-error-expdisc}
        \end{figure}
    }
    \only<5>{Is it solely the total number of neurons, or does the architecture have an impact?
    \begin{figure}
        \centering
        \includegraphics[width=0.65\textwidth]{imgs/tot-neurons-error.png}
        %\caption{Evolution of the average relative error as a function of the total number of neurons in the PINNs.}
        \label{fig:tot-neurons-error}
    \end{figure}
    }
\end{frame}


%include{chapters/conclusion}

\begin{frame}{Future Works}
    \begin{itemize}
        \item Jeans equations
        \item Influence of hyperparameters
        \item Theory and interpretability
        \item Inverse problem for parameters estimation
    \end{itemize}
\end{frame}
\bibliographystyle{apalike}
\bibliography{references}

\end{document}
