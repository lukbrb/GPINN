\chapter{Conclusion}\label{ch:conclusion}

In this trailblazing study, we have taken the initial stride towards the development of novel tools for simulating galaxies. The proficiency of Physics-Informed Neural Networks (PINNs) in solving the gravitational Poisson equation~\eqref{eq:poisson_eq_general} has been substantiated for three distinct density profiles, two of which are spherically symmetric and one that is axisymmetric. The initial two models yield comparatively accurate results - averaging 1.71\% and 3.75\% respectively - without necessitating any fine-tuning. Further research could potentially diminish the error of these PINNs, however, given that these two models admit an analytical solution, we did not undertake to optimize them in this particular study. The Poisson equation that describes the exponential thick disc model, on the other hand, does not admit an analytical solution, and hence the associated PINN can serve a genuine practical utility.

To confirm that the error could be considerably reduced, we conducted a rudimentary grid search, according to the parameters displayed in Table~\ref{tab:fine-tuning}. This grid search allowed us to select a PINN model that delivers an average relative error of 0.36\%, with a maximum of merely 0.99\%. These results are obtained for a fixed value of $\eta$ (see equation~\eqref{eq:poisson-exp-disc-final}), a ratio of the model's scale lengths. Despite the success of the two-parameter PINN, it remains to be demonstrated that the extension of this PINN to three dimensions performs as satisfactorily. This extension is an excellent opportunity to underscore once again the adaptability of PINNs. Utilizing techniques such as the finite difference method, it is not straightforward to modify the code to include an additional dimension. With a PINN, however, one merely needs to adjust the input data or possibly the PDE residual.


\section{Outlooks and Future Works}

As previously mentioned, this work is innovative in many respects, and while we have taken the initial step by demonstrating the capability of PINNs to solve certain gravitational potentials, there is still much to be accomplished. In the domain of galaxy modeling, one could first extend the work performed to other non-analytical models, or attempt to solve the Jeans equations, which are used to describe the velocity field of a galaxy. More generally, within the field of PINNs, there are several uncharted paths to investigate. We discuss a few of these in this section.
\subsection{Hyperparameters}
While the Hernquist~\cite{hernquist_analytical_1990} and Dehnen~\cite{dehnen_family_1993} profiles admit analytical solutions, it would nonetheless be necessary to undertake an extensive fine-tuning of the hyperparameters to ascertain whether the PINN can achieve an arbitrarily high precision or whether it is bounded. Additionally, the search for hyperparameters for the exponential thick disc allowed us to study the influence of these parameters on the PINN error (see Figures~\ref{fig:hyperparametres-vs-error} and~\ref{fig:tot-neurons-error}). This empirical exploration enables us to note that in the case of the exponential thick disc, the tanh function generally performs significantly better than the sigmoid or SiLU functions. We also observe that the network width-the number of neurons per layer-seems to cease improving the precision beyond a certain threshold. It appears that for an equivalent number of neurons, it is preferable to increase the network depth, as illustrated in Figure~\ref{fig:tot-neurons-error}. Ultimately, although literature~\cite{he_physics-informed_2020} suggests that the L-BFGS-B optimizer is the most effective under conditions similar to ours, it was unable to solve the Poisson equation for the Hernquist model. 
\par More broadly, a dedicated and rigorous exploration of hyperparameters would be beneficial and could potentially contribute to a deeper understanding of the operational intricacies of PINNs.

\subsection{Theory and Interpretability}
The empirical findings discussed above illustrate the challenges encountered with PINNs. Indeed, there are very few theoretical results that provide bounds on the error made by the PINN. A notable result is that of~\cite{de_ryck_approximation_2021}, which provides a bound on the PINN error as a function of the total number of neurons in the network. In another more recent paper~\cite{de_ryck_error_2023}, the same authors demonstrate that there exists a neural network approximating the classical solution of the Navier-Stokes equation such that the generalization error and the training error of the PINN are arbitrarily small. Explicit bounds on the number of neurons and network weights are also provided, depending on the error tolerance and the Sobolev norms of the underlying Navier-Stokes equation. However, these results are limited to two-layer networks using the tanh activation function and therefore do not elaborate on the influence of network width or depth in relation to the total number of neurons in the network. Ultimately, new theoretical results would be beneficial and necessary to prevent PINNs from remaining somewhat of a black box that cannot be fully deciphered.

\subsection{Inverse Problem}

While in this study we have demonstrated the proficiency of PINNs as solvers for swiftly simulating intricate systems, thereby allowing them to be used for parameter estimation, we have not yet fully harnessed the power of PINNs. Indeed, as outlined in Chapter~\ref{ch:pinns}, when furnished with simulation data of a system, a PINN possesses the capability to discern the parameters of a PDE. This is the so-called Inverse Problem. Therefore, if the end goal is to utilize the solution from a PINN over a given domain for parameter estimation, it naturally prompts the question as to whether it is feasible to utilize the inverse problem for ascertaining the parameters of a potential-density pair.