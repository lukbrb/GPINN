\chapter{Basics of Galactic Dynamics}\label{ch:galaxies-theory}
Galactic dynamics is a branch of astrophysics that studies the distribution of masses and their movements in galaxies. Galaxies are primarily composed of stars, gas, dust, and dark matter. Modeling the dynamics of galaxies allows for a better understanding of the evolution, formation, and interactions between galaxies. In this chapter, we introduce some basic concepts about gravitational fields, and in particular, the Poisson equation, which will guide the solution to the problems we will address in Chapter~\ref{ch:applications}. Finally, we  discuss spherical symmetry profiles and axisymmetric profiles.

\section{Gravitational Potential}\label{sec:gravity}
Gravity is the dominant force governing movements on the scale of galaxies. Particularly, most of a galaxy's mass is found in the stars composing it. To determine the total gravitational field of the galaxy, one could imagine summing the contribution of each of these stars.
\par However, this method is not feasible in practice due to the large number of stars\footnote{A typical galaxy contains about $10^{11}$ stars.} in a typical galaxy. For most cases, it is sufficient to smooth the mass density of the stars on a scale small compared to the size of the galaxy, but large compared to interstellar distance~\cite{binney2011galactic}.

\subsection{General Results}

To determine how we will compute the gravitational field given the mass density, we first look at the basic theory. Initially, the goal is to compute the force $\Vec{F}(x)$ exerted by the gravitational attraction generated by a mass distribution $\rho(\Vec{x'})$ on a particle of mass $m_s$ at position $\Vec{x}$. This force is obtained by summing all the small contributions $\delta\Vec{F}(x)$, defined as
\begin{equation}
\label{eq:fg_small_contrib}
\delta\Vec{F}(x) = Gm_s \cdot \frac{\Vec{x'} - \Vec{x}}{|\Vec{x'} - \Vec{x}|^3}\delta m(\Vec{x'}) = Gm_s \cdot \frac{\Vec{x'} - \Vec{x}}{|\Vec{x'} - \Vec{x}|^3}\rho(\Vec{x'}) \dd^3 \Vec{x'}\text{ ,}
\end{equation}
to the global force generated by each volume element $\dd^3 x'$ located at $\Vec{x'}$. This is written as follows:
\begin{equation}
\label{eq:fg_def}
\Vec{F}(x) = m\Vec{g}(x) \text{ where } \Vec{g}(x) \equiv G \int \dd^3 x' \frac{\Vec{x'} - \Vec{x}}{|\Vec{x'} - \Vec{x}|^3}\rho(\Vec{x'})\text{,}
\end{equation}
where $G$ is naturally the gravitational constant.
The gravitational field $\Vec{g}(x)$ is related to the gravitational potential $\Phi(x)$ by the following relation:
\begin{equation}
\label{eq:gfield_to_phi}
\Vec{g}(x) = - \Vec{\nabla}_x \Phi(x)
\end{equation}

The potential is convenient because it is a scalar field, easier to visualize and manipulate than the gravitational field, but containing the same information.

\subsection{Poisson's Equation}

By taking the divergence of the gravitational field $\Vec{g}$, we can also show that all contributions must come from the point $x=x'$. Therefore, we can restrict the integration volume to a small sphere of radius $h$ centered at point $x$. Since $h$ is small, we can consider $\rho(\Vec{x})$ constant across the volume of the sphere. This allows us to extract it from the following integral~\cite{binney2011galactic}:

\begin{equation}
    \begin{split}
    \label{eq:demo_eq_poisson}
    \Vec{\nabla}\cdot \Vec{g}(x) &= G\rho (x) \int_{r \leq h} \dd^3 x' \Vec{\nabla}_x \cdot \left(\frac{x'-x}{|x'-x|^3}\right)\\
                                 &= - G\rho (x) \int_{r \leq h} \dd^3 x' \Vec{\nabla}_x' \cdot \left(\frac{x'-x}{|x'-x|^3}\right)\\
                                 &= - G\rho (x) \int_{r = h} \dd^2 \Vec{S}'\cdot \left(\frac{x'-x}{|x'-x|^3}\right)\\
    \end{split}
\end{equation}
where we have set $r=|x'-x|$ to lighten the notation. Since at the surface of a sphere we have $\dd^2 \Vec{S} = h (x'-x)  \dd^2 \Omega$, Equation~\eqref{eq:demo_eq_poisson} becomes:
\begin{equation}
    \label{eq:demo_poisson2}
    \Vec{\nabla}\cdot \Vec{g}(x) = - G\rho (x) \int \dd^2 \Omega = -4\pi G\rho(\Vec{x})
\end{equation}

Finally, by substituting Equation~\eqref{eq:gfield_to_phi} into Equation~\eqref{eq:demo_poisson2}, we arrive at Poisson's equation, which links the gravitational potential to the mass distribution density $\rho(\Vec{x})$:

\begin{equation}
\label{eq:poisson_eq_general}
\nabla^2 \Phi(\Vec{x}) = 4 \pi G \rho(\Vec{x})
\end{equation}

\par Poisson's equation is a differential equation that provides $\Phi(\Vec{x})$ given $\rho(\Vec{x})$ and appropriate boundary conditions. In our context, we will consider isolated systems for which $\Phi \rightarrow 0$ as $|\Vec{x}| \rightarrow \infty$.

\section{Spherically Symmetric Profiles}\label{sec:spherical-symmetry}
Spherically symmetric profiles are a simplified approximation suitable for certain galaxies. In such cases, the mass distribution and the potential depend only on the radial distance $r$ from the galaxy center, and the Laplacian in Poisson's equation simplifies to:

\begin{equation}
\label{eq:poisson_spherique}
\nabla^2 \Phi(r) = \frac{1}{r^2}\frac{\dd}{\dd r}\left(r^2\frac{\dd \Phi}{\dd r}\right)
\end{equation}

Solving Poisson's equation for spherical profiles depends on the specific form of the mass distribution density $\rho(r)$. We will delve into the distributions of Hernquist~\cite{hernquist_analytical_1990} and Dehnen~\cite{dehnen_family_1993} profiles.

\subsection{Hernquist Profile}
The observed luminosity distribution of numerous galactic bulges and elliptical galaxies is well characterized by the empirical law
\begin{equation}
\label{eq:deVaucouleur}
\log_{10} \left[ \frac{I(R)}{I(R_e)}\right] = -3.331 \left[ \left(\frac{R}{R_e}\right)^{1/4} - 1\right]\text{,}
\end{equation}
where $R$ is the radius projected onto the plane of the sky, $R_e$ is the effective radius of the isophote enclosing half the light, and $I$ is the surface brightness~\cite{deVaucouleurs1948}. The Hernquist potential-density pair~\cite{hernquist_analytical_1990} retains the same properties as Equation~\eqref{eq:deVaucouleur}, but allows for analytical solutions. The density proposed by Hernquist is:

\begin{equation}
\label{eq:hernquist_density}
\rho(r) = \dfrac{M}{2\pi}\dfrac{a}{r}\dfrac{1}{(r+a)^3}\text{,}
\end{equation}
where $M$ is the total mass of the system and $a$ is a scalelength. For such a density, Poisson's Equation~\eqref{eq:poisson_spherique} takes the following form:

\begin{equation*}
\dfrac{1}{r^2} \dfrac{\partial}{\partial r}\left(r^2 \dfrac{\partial \Phi}{\partial r}\right) = 4\pi G \left[\dfrac{M}{2\pi}\dfrac{a}{r}\dfrac{1}{(r+a)^3}\right]
\end{equation*}
Given that the potential-density pair depends only on the radial coordinate $r$, the previous equation can be rewritten as
\begin{equation*}
\dfrac{1}{r} \dfrac{\dd}{\dd r}\left(r^2 \dfrac{\dd \Phi}{\dd r}\right) = \dfrac{2GMa}{a^3(\frac{r}{a}+1)^3}
\end{equation*}

To later facilitate the computation of this potential, we aim to make this equation dimensionless. We thus set $s = \frac{r}{a}$ and $\Phi'(r) = \Phi(r)/\frac{GM}{a}$. We then get the dimensionless Poisson equation in the case of Hernquist's density:

\begin{equation}
\label{eq:residual_poisson_hernquist1}
\boxed{\dfrac{\dd}{\dd s}\left(s^2 \dfrac{\dd \Phi'}{\dd s}\right) = \dfrac{2s}{(s+1)^3}}
\end{equation}

The solution to this equation can be calculated analytically (see Appendix~\ref{app:poisson}), resulting in:

\begin{equation}
\label{eq:hernquist_pot}
\Phi(s) = - \frac{1}{s + 1}
\end{equation}

\subsection{Dehnen Profile}
Dehnen's profiles~\cite{dehnen_family_1993} are a family of potential-density pairs describing spherical galaxies and bulges. The model includes an additional parameter, $\gamma \in [0, 3[$ representing the inner slope of the model. Hernquist's~\cite{hernquist_analytical_1990} and Jaffe's~\cite{jaffe1983simple} models are included in Dehnen's family of potential-density pairs as special cases when $\gamma=1$ and $\gamma=2$, respectively. Finally, in this case, the density can be written as follows:

\begin{equation}
    \label{eq:density-dehnen}
    \rho(r) = \dfrac{(3-\gamma)M}{4\pi}\dfrac{a}{r^{\gamma}(r+a)^{4-\gamma}}\text{,}
\end{equation}

Performing the same procedure as for the Hernquist model, it can be easily shown that the associated dimensionless Poisson equation can be written as:

\begin{equation}
\label{eq:poisson-dehnen}
\dfrac{\dd}{\dd s}\left(s^2 \dfrac{\dd \Phi'}{\dd s}\right) = \dfrac{2s^{2-\gamma}}{(1+s)^{4-\gamma}}
\end{equation}
which has the following solution~\cite{dehnen_family_1993}:

\begin{equation}
    \label{eq:pot-dehnen}
    \Phi(s) = 
    \begin{cases} 
    -\dfrac{1}{2 - \gamma} \left[1 - \left( \dfrac{s}{1 + s} \right )^{2-\gamma}\right] & \text{if } \gamma \neq 2 \\
    \\
    \ln \left(\dfrac{s}{1 + s}\right) & \text{if } \gamma = 2
    \end{cases}
\end{equation}

\section{Axisymmetric Profiles}\label{sec:axisymmetry}

Axisymmetric profiles offer a better approximation for galaxies that exhibit more complex shapes, such as spiral galaxies. In this case, the mass distribution depends on both the radius $R$ in the equatorial plane and the vertical coordinate $z$ perpendicular to this plane. The Laplacian in the Poisson equation then takes the following form:

\begin{equation}
\label{eq:poisson_axisymmetrique}
\nabla^2 \Phi(R, z) = \frac{1}{R}\frac{\partial}{\partial R}\left(R\frac{ \partial \Phi}{\partial R}\right) + \frac{\partial^2\Phi}{\partial z^2}
\end{equation}

Certain axisymmetric profiles admit analytical solutions, such as the Kuzmin profile or the Miyamoto-Nagai profile~\cite{miyamoto1975three}. More generally, solving the Poisson equation for this class of profiles requires numerical methods. In particular, we will be interested in Section~\ref{sec:disk} in the model of the thick exponential disk, which does not admit an analytical solution.

\subsection{Exponential Disk}

The mass distribution of the stellar disk of most galaxies is well represented by an exponential radial profile~\cite{freeman1970disks}

\begin{equation*}
\Sigma(R) = \Sigma_0 \exp{\left(-\frac{R}{R_d}\right)}\text{,}
\end{equation*}
where $\Sigma$ is the surface density, $\Sigma_0$ the surface density at the center of the disk, $R$ is the radius within the disk, and $R_d$ is a scalelength of the disk. These disks can have variable vertical distributions, which are commonly modeled with a hyperbolic secant function $\text{sech}^n = \cosh^{-n}$. Thus, to fully describe galaxy disks, the following formula is used~\cite{smith2015simple}:

\begin{equation}
\label{eq:exp_disk_general}
\rho(R, z) = \rho_0 \exp{\left(-R/R_d\right)}\cdot \text{sech}^n\left(-|z|/z_d\right)\text{,}
\end{equation}
where $z_d$ is a  scaleheight and $n$ is typically between  $\sim 1$ and $3$. $\rho_0$ is the central mass density of the disk, and is usually normalized by the scalelengths of the model. We note the special case $n \rightarrow \infty$ which describes the doubly exponential disk; the vertical component also decreases exponentially.

\subsection{Thick Exponential Disk}

In our study, we are interested in the case of the thick exponential disk. Particularly we study the particular case when $n=2$. Equation~\eqref{eq:exp_disk_general} thus becomes:

\begin{equation}
    \label{eq:exp-disk}
    \rho(R, z) = \rho_0 \exp{\left(-R/R_d\right)}\cdot \cosh^{-2}{\left(-|z|/z_d\right)}
\end{equation}

Firstly, we want to find the associated dimensionless Poisson equation, and then we will look at the approximate solution of this equation~\eqref{eq:exp-disk-potential}. Using the Laplacian from equation~\eqref{eq:poisson_axisymmetrique}, we get directly:

\begin{equation}
\label{eq:exp-disc-poisson1}
\dfrac{1}{R} \dfrac{\partial}{\partial R} \left( R \dfrac{\partial \Phi}{\partial R}\right) + \dfrac{\partial^2 \Phi}{\partial z^2} = 4\pi G \rho_0 \exp{\left(-R/R_d\right)}\cdot \cosh^{-2}{\left(-|z|/z_d\right)}
\end{equation}
Now, by setting $z' = z/z_d$ and $R' = R/R_d$, we can rewrite~\eqref{eq:exp-disc-poisson1} in the form:

\begin{equation*}
\dfrac{1}{R_{d}^{2}} \dfrac{1}{R'} \dfrac{\partial}{\partial R'} \left(R' \dfrac{\partial \Phi}{\partial R'}\right) + \dfrac{1}{z_{d}^{2}}\dfrac{\partial^2 \Phi}{\partial z'^2} = 4\pi G \rho_0 \exp{-R'} \cosh^{-2}{z'}
\end{equation*}
Finally, if we set $\eta = z_d/R_d $ and $\phi'= \frac{\phi}{G M_d/z_d}$, we get a dimensionless Poisson equation for the thick exponential disk:

\begin{equation}
\label{eq:poisson-exp-disc-final}
\dfrac{1}{R'} \dfrac{\partial}{\partial R'} \left(R' \dfrac{\partial \Phi'}{\partial R'}\right) + \dfrac{1}{\eta^{2}}\dfrac{\partial^2 \Phi'}{\partial z'^2} = e^{-R'} \cosh^{-2}{z'}
\end{equation}

In order to implement a numerical solution for this equation, we follow the results of Appendix A of~\cite{bonetti2021dynamical}. The potential is given by the following equation:

\begin{equation}
\label{eq:exp-disk-potential}
\Phi(R, z) = - 2\pi G \alpha \rho_0 \int_{0}^{\infty} dk J_0 (kR) \dfrac{I_z (k)}{(\alpha^2 + k^2)^{\frac{3}{2}}}\text{,}
\end{equation}
with
\begin{equation*}
I_z(k) = \dfrac{4}{\beta} \left\{ 1 - \dfrac{k}{k+\beta} \left[ e^{-z\beta} {}_2F_1\left(1, 1 + \frac{k}{\beta}; 2 + \frac{k}{\beta}; -e^{-z\beta} \right) + e^{z\beta} {}_2F_1\left(1, 1 + \frac{k}{\beta}; 2 + \frac{k}{\beta}; -e^{z\beta} \right)\right] \right\}
\end{equation*}
where $_2F_1$ is the Gaussian hypergeometric function, defined such that :

\begin{equation}
    \label{eq:hypergeometric}
    _2F_1(a, b, c; z) = \sum_{n=0}^{\infty} \frac{(a)_n (b)_n}{(c)_n}\cdot \dfrac{z^n}{n!}
\end{equation}
where $(a)_n$ is the Pochhammer symbol and $|z| < 1$. Note that we also set $\alpha=1/R_d$ and $\beta = 2/z_d$.


