\documentclass[a4paper, 11pt, twoside]{report}
\usepackage[a4paper]{geometry}
\usepackage{lmodern}
\usepackage{textcomp}
\usepackage{url}
\usepackage{float}
\usepackage{booktabs}
\usepackage{longtable}
\usepackage{makeidx}
\usepackage{fancyhdr}
\usepackage[times]{quotchap}
\usepackage{tikz}
\usepackage{multirow}
\usepackage{version}
\usepackage{tcolorbox}
\usepackage{listings}
\usepackage{xcolor}
\usepackage{hyperref}
\usepackage[T1]{fontenc}
\usepackage{graphicx} % Required for inserting images
\usepackage{hyperref}
\usepackage{amsmath, amsfonts}
\usepackage{apalike}
\usepackage{mathtools}
\usepackage{minted}
\usepackage{caption}
\usepackage{subcaption}
\usepackage{rotating}
%\usepackage[acronym]{glossaries-extra}

\newcommand{\dd}{\mathrm{d}}
\renewcommand{\L}{\mathcal{L}}
\renewcommand{\Vec}{\mathbf}
\newcommand{\mytitle}{Application of Physics-Informed Neural Networks for Galaxy Dynamics}
\newcommand{\myname}{Lucas Barbier-Goy}
\newcommand{\mysupervisor}{Prof. Dr. Marco Landoni \& Dr. Fabio Rigamonti}
\newcommand{\mydate}{May 25\textsuperscript{th}}

\hypersetup{
    colorlinks=false,
    linkcolor=blue,
    urlcolor=blue,
    citecolor=black,
    pdftitle={Application of Physics-Informed Neural Networks for Galaxy Dynamics} 
    }


%% Gestione header: no header sulle dispari bianche
\makeatletter
\def\cleardoublepage{\clearpage\if@twoside \ifodd\c@page\else%
    \hbox{}%
    \thispagestyle{empty}%              % Empty header styles
    \newpage%
    \if@twocolumn\hbox{}\newpage\fi\fi\fi}
\makeatother


\makeindex
\linespread{1.1}



%% Aggiunge una linea al di sotto di ogni sezione principale
\usepackage[calcwidth]{titlesec}
\titleformat{\section}[hang]{\sffamily\bfseries}
 {\Large\thesection}{12pt}{\Large}[{\titlerule[0.4pt]}]

 %% Gestione header: no header sulle dispari bianche
\makeatletter
\def\cleardoublepage{\clearpage\if@twoside \ifodd\c@page\else%
    \hbox{}%
    \thispagestyle{empty}%              % Empty header styles
    \newpage%
    \if@twocolumn\hbox{}\newpage\fi\fi\fi}
\makeatother

\begin{document}

% \maketitle
% FRONT PAGE -------------------------------------------------------------------
 
\begin{titlepage}
\begin{center}
    
\LARGE
Master's Thesis
    
\vspace{0.5cm}
      
\rule{\textwidth}{1.5pt}
\LARGE
\textbf{\mytitle}
\rule{\textwidth}{1.5pt}
   
\vspace{0.5cm}
      
\large
Dipartimento di Scienza e Alta Tecnologia \\
Università degli Studi dell'Insubria 

\vfill

\Large
\textbf{\myname}

\vfill

\large
Como, \mydate, 2023.
      
\vfill

\includegraphics[width = 0.4\textwidth]{imgs/logo-insubria.pdf}

\vfill

\normalsize
%Submitted in partial fulfillment of the requirements for the degree of M. Sc.

Supervised by \mysupervisor \\
Co-supervisor: Prof. Dr. Carlo Canali | 
Examiner: Assoc. Prof. Dr. Magnus Paulsson

\end{center}
\end{titlepage}

\begin{titlepage}
\begin{center}
    
\LARGE
Master's Thesis
    
\vspace{0.5cm}
      
\rule{\textwidth}{1.5pt}
\LARGE
\textbf{\mytitle}
\rule{\textwidth}{1.5pt}
   
\vspace{0.5cm}
      
\large
Institutionen för fysik och elektroteknik \\
Linnéuniversitetet 

\vfill

\Large
\textbf{\myname}

\vfill

\large
Kalmar, \mydate, 2023.
      
\vfill

\includegraphics[width = 0.45\textwidth]{imgs/logo-lnu.png}

\vfill

\normalsize
%Submitted in partial fulfillment of the requirements for the degree of M. Sc.

Supervised by \mysupervisor \\
Co-supervisor: Prof. Dr. Carlo Canali | 
Examiner: Assoc. Prof. Dr. Magnus Paulsson

\end{center}
\end{titlepage}
\pagenumbering{roman}
\setcounter{page}{1}
\setcounter{tocdepth}{2}

%\begin{abstract}
%Developing efficient and accurate numerical methods to simulate dynamics of physical systems has been an everlasting challenge in computational physics. Physics-Informed Neural Networks are neural networks that encode laws of physics in their structure. Using auto-differentiation, they can solve partial differential equations (PDE) by minimizing the loss function at some points of the domain of interest. The efficiency attained by these networks for solving PDEs place them as ideal solvers for simulating complex systems.
%\par In this novel work, we take a first step towards simulating galaxy dynamics with PINNs by solving the gravitational Poisson equation. We first verify the capacity of PINNs to solve the gravitational Poisson equation for simple radial density profiles. We then extend the study to a more complex axisymmetric density profile, making the PINN a function of two parameters. Fine-tuning the We show clear advantages of PINNs over regular solvers in terms of efficiency.
%\end{abstract}

\begin{abstract}
Developing efficient and accurate numerical methods to simulate dynamics of physical systems has been an everlasting challenge in computational physics. Physics-Informed Neural Networks (PINNs) are neural networks that encode laws of physics into their structure. Utilizing auto-differentiation, they can efficiently solve partial differential equations (PDEs) by minimizing the loss function at certain points within the domain of interest. The remarkable efficiency exhibited by these networks when solving PDEs positions them as ideal solvers for simulating complex systems.
\par
In this pioneering work, we take a first step towards simulating galaxy dynamics using PINNs by solving the gravitational Poisson equation. We initially substantiate the capacity of PINNs to solve the gravitational Poisson equation for the simple Hernquist~\cite{hernquist_analytical_1990} radial density profile, and for the parametric Dehnen~\cite{dehnen_family_1993} radial density profile. Following this, we extended our study to encompass a more complex axisymmetric density profile describing a Thick Exponentiel Disk.

The capacity of PINNs to generate comparatively accurate results has been validated with an average error of 1.71\% and 3.75\% respectively for the spherically symmetric Hernquist and Dehnen models. While for the axisymmetric thick exponentiel disk model the PINN demonstrated an average relative error of 0.36\% with a maximum error of just 0.99\% after fine-tuning the PINN's hyperparameters. Although this model typically relies on the two coordinates $R$ and $z$ along with the ratio $\eta$ of the model's scale lengths, the PINN is here trained using a fixed, predetermined value of $\eta$. 

Drawing upon the outcomes of the grid search implemented for the thick exponential disk model, we provide a succinct examination of how the hyperparameters of the PINN impact the relative error. Given the limited quantity of datapoints, we refrain from formulating definitive conclusions, yet we do exhibit certain discernible patterns. Specifically, we demonstrate that the hyperbolic tangent (tanh) activation function consistently outperforms other activation functions in the context of our model. Additionally, it appears that augmenting the depth of the network offers superior error reduction in comparison to increasing its width, reinforcing the importance of architectural considerations in the optimization of Physics-Informed Neural Networks

Our results show clear advantages of PINNs over regular solvers in terms of efficiency. Despite the success of the two-parameter PINN for the thick exponential disk, further work is required to confirm its extension to three dimensions. This pioneering research offers a promising foundation for further developments in the field, and demonstrates the genuine practical utility of PINNs for simulating complex systems such as galaxies.
\end{abstract}

\newpage
\section*{Acknowledgment}

I wish to express profound gratitude to my supervisors Marco Landoni and Fabio Rigamonti for presenting me with this  subject, and for their unyielding encouragement, unwavering guidance, and insightful feedback which facilitated my progression and realization of this thesis. I owe a great deal to Marco for his guidance, which helped me stay focused throughout the project. Furthermore, I am indebted to Fabio, whose enthusiasm, interest, and dedication to this work have guided me throughout this study.

\tableofcontents
\listoffigures
\listoftables


        \pagestyle{fancy}
        \renewcommand{\chaptermark}[1]{\markboth{#1}{}} 
        \renewcommand{\sectionmark}[1]{\markright{\thesection\ #1}} 
        \fancyhf{} % delete current setting for header and footer 
        \fancyhead[LE,RO]{\bfseries\thepage} 
        \fancyhead[LO]{\bfseries\rightmark} 
        \fancyhead[RE]{\bfseries\leftmark} 
        \renewcommand{\headrulewidth}{0.8pt} 
        \renewcommand{\footrulewidth}{0pt} 
        %\renewcommand{\headheight}{13.59999pt}
        \addtolength{\headheight}{0.5pt} % make space for the rule 
        \fancypagestyle{plain}{% 
        \fancyhead{} % get rid of headers on plain pages
        \fancyfoot[C]{\bfseries \thepage}
        \renewcommand{\headrulewidth}{0pt} % and the line 
        } 

        \cleardoublepage{}
        \pagenumbering{arabic}
        \setcounter{page}{1}
\chapter{Introduction}\label{ch:intro}
In the realm of galactic dynamics, the overarching task of calculating the total potential of the galaxy could, in theory, be attained by summing the potentials of punctual masses corresponding to all the stars, given that the majority of the galaxy's mass resides in these stellar bodies. A typical galaxy boasts approximately $10^{11}$ stars, rendering the latter proposition largely unfeasible~\cite{binney2011galactic}. For the dynamic modeling of a galaxy, it generally suffices to accurately depict the galaxy's density field and to compute its gravitational potential via the Poisson equation. As is often the case in physics, analytical solutions—here, analytical potential-density pairs—are available for simplistic systems, yet in more realistic scenarios, numerical quadrature becomes a necessity~\cite{caravita_jeans_2021}. Numerical solutions frequently entail time-consuming calculations, especially when real galaxy dynamic modeling and Bayesian parameter estimation methods are in play~\cite{rigamonti2022maximally}.

Artificial Intelligence (AI) has ascended to become an indispensable part of modern scientific research across numerous domains. This ubiquity is largely justified by the enormous volume of data presently accessible. However, there exist fields wherein data quantity is limited. In such instances, one would aspire to leverage AI techniques to utilize the available data. An ingenious solution to this issue was proposed by~\cite{raissi_physics_2017}. The initial idea is to exploit the properties of neural networks (NNs) as universal function approximators, taking advantage of their auto-differentiation capability. Subsequently, the knowledge we possess in physics (symmetry, invariance, or conservation principles originating from the physical laws governing the observed data) is utilized to constrain the parameter space during the training phase of the neural network. This is referred to as Physics-Informed Neural Networks (PINNs). These PINNs can be utilized to solve nonlinear partial differential equations (PDEs) encountered in diverse physics problems—such as advection problems, diffusion problems, flow problems, etc. The PINNs can address a broad range of issues in computational sciences and leads to the introduction of new classes of numerical solvers for PDEs. Similar to grid-based simulations with adaptive mesh refinement, resolving the Poisson equation via the multigrid approach or Fourier methods is quite costly, the PINNs could potentially induce a paradigm shift in the study of complex systems by pushing hardware and computational boundaries.

The principal objective of this thesis is to enhance galactic modeling employing PINNs. We will study specific density profiles, commencing with simple analytical density profiles (see Section~\ref{sec:spherical-symmetry}), to then transition to more intricate, axisymmetric density profiles that are closer to reality (Section~\ref{sec:axisymmetry}). This endeavor could enable researchers working on galactic dynamics to perform parameter estimates with considerable time savings, opening previously untraversable paths. Other applications of PINNs in astrophysics exist, in a similar vein, they could be utilized to expedite the creation of initial conditions for galaxy simulations or complex system simulations.

We will lean on recent developments of PINNs to resolve the Poisson equation for the gravitational potential~\cite{kharazmi_variational_2019}. As a proof of concept, we will initially test the Hernquist density profile (see Section~\ref{sec:hernquist}). In this case, the gravitational potential can be calculated analytically and can be used to test and validate the accuracy of the PINNs. Once the ability of PINNs to resolve the Poisson equation is substantiated, we will attempt to resolve the potential for the Dehnen density profile (see Section~\ref{sec:dehnen}). While the Hernquist profile carries two parameters, mass and scale radius, the Dehnen profile has three: mass, scale radius, and the inner slope, denoted $\gamma$. The Dehnen profile has an analytical solution, but the Poisson equation now depends on an additional parameter, $\gamma$. With this second profile, we intend to assess the accuracy of PINNs in solving parametric differential equations.

Ultimately, we will proceed with more realistic, axisymmetric density profiles, such as the Thick Exponential Disk (Section~\ref{sec:disk}). Depending on the success of the latter, we could extend the project by applying the same technique to solve the Jeans equation of gravity, commonly used to predict galaxy velocity fields.

From the right choice of network architecture, to smart tricks for achieving convergence, or simplifying equations, the challenges faced in this project are numerous. The choice of architecture for the physics-informed neural networks is still an ongoing research topic, and could be the most demanding task. It's noteworthy to mention that to the best of our knowledge, no work has been done on solving the gravitational Poisson equation by PINNs.

This thesis is organized in six chapters. In Chapter~\ref{ch:galaxies-theory} is presented the theory behind galaxy dynamics, and particularly the Poisson equation applied to some models of interest. Chapter~\ref{ch:neural-networks} aims to give an overview of what neural networks are. Are also detailed some theorem or algorithms that justify the use of neural networks for approximating a gravitational potential. A definition of PINNs and their characteristics is given in Chapter~\ref{ch:pinns}. Finally, in Chapter~\ref{ch:applications} are presented the method with which we have used PINNs as well as the results obtained. We conclude in Chapter~\ref{ch:conclusion}. 

\par Also, as triviality is a vague concept that the best AI models could not define, the equations and calculations are detailed as much as possible for my fellow students who wish to go through this thesis. Some theorem or calculations can be found in the Appendices, and all the code used for this work is available on the following GitHub repository: \url{https://github.com/lukbrb/GPINN}  

\chapter{Basics of Galactic Dynamics}\label{ch:galaxies-theory}
Galactic dynamics is a branch of astrophysics that studies the distribution of masses and their movements in galaxies. Galaxies are primarily composed of stars, gas, dust, and dark matter. Modeling the dynamics of galaxies allows for a better understanding of the evolution, formation, and interactions between galaxies. In this chapter, we introduce some basic concepts about gravitational fields, and in particular, the Poisson equation, which will guide the solution to the problems we will address in Chapter~\ref{ch:applications}. Finally, we  discuss spherical symmetry profiles and axisymmetric profiles.

\section{Gravitational Potential}\label{sec:gravity}
Gravity is the dominant force governing movements on the scale of galaxies. Particularly, most of a galaxy's mass is found in the stars composing it. To determine the total gravitational field of the galaxy, one could imagine summing the contribution of each of these stars.
\par However, this method is not feasible in practice due to the large number of stars\footnote{A typical galaxy contains about $10^{11}$ stars.} in a typical galaxy. For most cases, it is sufficient to smooth the mass density of the stars on a scale small compared to the size of the galaxy, but large compared to interstellar distance~\cite{binney2011galactic}.

\subsection{General Results}

To determine how we will compute the gravitational field given the mass density, we first look at the basic theory. Initially, the goal is to compute the force $\Vec{F}(x)$ exerted by the gravitational attraction generated by a mass distribution $\rho(\Vec{x'})$ on a particle of mass $m_s$ at position $\Vec{x}$. This force is obtained by summing all the small contributions $\delta\Vec{F}(x)$, defined as
\begin{equation}
\label{eq:fg_small_contrib}
\delta\Vec{F}(x) = Gm_s \cdot \frac{\Vec{x'} - \Vec{x}}{|\Vec{x'} - \Vec{x}|^3}\delta m(\Vec{x'}) = Gm_s \cdot \frac{\Vec{x'} - \Vec{x}}{|\Vec{x'} - \Vec{x}|^3}\rho(\Vec{x'}) \dd^3 \Vec{x'}\text{ ,}
\end{equation}
to the global force generated by each volume element $\dd^3 x'$ located at $\Vec{x'}$. This is written as follows:
\begin{equation}
\label{eq:fg_def}
\Vec{F}(x) = m\Vec{g}(x) \text{ where } \Vec{g}(x) \equiv G \int \dd^3 x' \frac{\Vec{x'} - \Vec{x}}{|\Vec{x'} - \Vec{x}|^3}\rho(\Vec{x'})\text{,}
\end{equation}
where $G$ is naturally the gravitational constant.
The gravitational field $\Vec{g}(x)$ is related to the gravitational potential $\Phi(x)$ by the following relation:
\begin{equation}
\label{eq:gfield_to_phi}
\Vec{g}(x) = - \Vec{\nabla}_x \Phi(x)
\end{equation}

The potential is convenient because it is a scalar field, easier to visualize and manipulate than the gravitational field, but containing the same information.

\subsection{Poisson's Equation}

By taking the divergence of the gravitational field $\Vec{g}$, we can also show that all contributions must come from the point $x=x'$. Therefore, we can restrict the integration volume to a small sphere of radius $h$ centered at point $x$. Since $h$ is small, we can consider $\rho(\Vec{x})$ constant across the volume of the sphere. This allows us to extract it from the following integral~\cite{binney2011galactic}:

\begin{equation}
    \begin{split}
    \label{eq:demo_eq_poisson}
    \Vec{\nabla}\cdot \Vec{g}(x) &= G\rho (x) \int_{r \leq h} \dd^3 x' \Vec{\nabla}_x \cdot \left(\frac{x'-x}{|x'-x|^3}\right)\\
                                 &= - G\rho (x) \int_{r \leq h} \dd^3 x' \Vec{\nabla}_x' \cdot \left(\frac{x'-x}{|x'-x|^3}\right)\\
                                 &= - G\rho (x) \int_{r = h} \dd^2 \Vec{S}'\cdot \left(\frac{x'-x}{|x'-x|^3}\right)\\
    \end{split}
\end{equation}
where we have set $r=|x'-x|$ to lighten the notation. Since at the surface of a sphere we have $\dd^2 \Vec{S} = h (x'-x)  \dd^2 \Omega$, Equation~\eqref{eq:demo_eq_poisson} becomes:
\begin{equation}
    \label{eq:demo_poisson2}
    \Vec{\nabla}\cdot \Vec{g}(x) = - G\rho (x) \int \dd^2 \Omega = -4\pi G\rho(\Vec{x})
\end{equation}

Finally, by substituting Equation~\eqref{eq:gfield_to_phi} into Equation~\eqref{eq:demo_poisson2}, we arrive at Poisson's equation, which links the gravitational potential to the mass distribution density $\rho(\Vec{x})$:

\begin{equation}
\label{eq:poisson_eq_general}
\nabla^2 \Phi(\Vec{x}) = 4 \pi G \rho(\Vec{x})
\end{equation}

\par Poisson's equation is a differential equation that provides $\Phi(\Vec{x})$ given $\rho(\Vec{x})$ and appropriate boundary conditions. In our context, we will consider isolated systems for which $\Phi \rightarrow 0$ as $|\Vec{x}| \rightarrow \infty$.

\section{Spherically Symmetric Profiles}\label{sec:spherical-symmetry}
Spherically symmetric profiles are a simplified approximation suitable for certain galaxies. In such cases, the mass distribution and the potential depend only on the radial distance $r$ from the galaxy center, and the Laplacian in Poisson's equation simplifies to:

\begin{equation}
\label{eq:poisson_spherique}
\nabla^2 \Phi(r) = \frac{1}{r^2}\frac{\dd}{\dd r}\left(r^2\frac{\dd \Phi}{\dd r}\right)
\end{equation}

Solving Poisson's equation for spherical profiles depends on the specific form of the mass distribution density $\rho(r)$. We will delve into the distributions of Hernquist~\cite{hernquist_analytical_1990} and Dehnen~\cite{dehnen_family_1993} profiles.

\subsection{Hernquist Profile}
The observed luminosity distribution of numerous galactic bulges and elliptical galaxies is well characterized by the empirical law
\begin{equation}
\label{eq:deVaucouleur}
\log_{10} \left[ \frac{I(R)}{I(R_e)}\right] = -3.331 \left[ \left(\frac{R}{R_e}\right)^{1/4} - 1\right]\text{,}
\end{equation}
where $R$ is the radius projected onto the plane of the sky, $R_e$ is the effective radius of the isophote enclosing half the light, and $I$ is the surface brightness~\cite{deVaucouleurs1948}. The Hernquist potential-density pair~\cite{hernquist_analytical_1990} retains the same properties as Equation~\eqref{eq:deVaucouleur}, but allows for analytical solutions. The density proposed by Hernquist is:

\begin{equation}
\label{eq:hernquist_density}
\rho(r) = \dfrac{M}{2\pi}\dfrac{a}{r}\dfrac{1}{(r+a)^3}\text{,}
\end{equation}
where $M$ is the total mass of the system and $a$ is a scalelength. For such a density, Poisson's Equation~\eqref{eq:poisson_spherique} takes the following form:

\begin{equation*}
\dfrac{1}{r^2} \dfrac{\partial}{\partial r}\left(r^2 \dfrac{\partial \Phi}{\partial r}\right) = 4\pi G \left[\dfrac{M}{2\pi}\dfrac{a}{r}\dfrac{1}{(r+a)^3}\right]
\end{equation*}
Given that the potential-density pair depends only on the radial coordinate $r$, the previous equation can be rewritten as
\begin{equation*}
\dfrac{1}{r} \dfrac{\dd}{\dd r}\left(r^2 \dfrac{\dd \Phi}{\dd r}\right) = \dfrac{2GMa}{a^3(\frac{r}{a}+1)^3}
\end{equation*}

To later facilitate the computation of this potential, we aim to make this equation dimensionless. We thus set $s = \frac{r}{a}$ and $\Phi'(r) = \Phi(r)/\frac{GM}{a}$. We then get the dimensionless Poisson equation in the case of Hernquist's density:

\begin{equation}
\label{eq:residual_poisson_hernquist1}
\boxed{\dfrac{\dd}{\dd s}\left(s^2 \dfrac{\dd \Phi'}{\dd s}\right) = \dfrac{2s}{(s+1)^3}}
\end{equation}

The solution to this equation can be calculated analytically (see Appendix~\ref{app:poisson}), resulting in:

\begin{equation}
\label{eq:hernquist_pot}
\Phi(s) = - \frac{1}{s + 1}
\end{equation}

\subsection{Dehnen Profile}
Dehnen's profiles~\cite{dehnen_family_1993} are a family of potential-density pairs describing spherical galaxies and bulges. The model includes an additional parameter, $\gamma \in [0, 3[$ representing the inner slope of the model. Hernquist's~\cite{hernquist_analytical_1990} and Jaffe's~\cite{jaffe1983simple} models are included in Dehnen's family of potential-density pairs as special cases when $\gamma=1$ and $\gamma=2$, respectively. Finally, in this case, the density can be written as follows:

\begin{equation}
    \label{eq:density-dehnen}
    \rho(r) = \dfrac{(3-\gamma)M}{4\pi}\dfrac{a}{r^{\gamma}(r+a)^{4-\gamma}}\text{,}
\end{equation}

Performing the same procedure as for the Hernquist model, it can be easily shown that the associated dimensionless Poisson equation can be written as:

\begin{equation}
\label{eq:poisson-dehnen}
\dfrac{\dd}{\dd s}\left(s^2 \dfrac{\dd \Phi'}{\dd s}\right) = \dfrac{2s^{2-\gamma}}{(1+s)^{4-\gamma}}
\end{equation}
which has the following solution~\cite{dehnen_family_1993}:

\begin{equation}
    \label{eq:pot-dehnen}
    \Phi(s) = 
    \begin{cases} 
    -\dfrac{1}{2 - \gamma} \left[1 - \left( \dfrac{s}{1 + s} \right )^{2-\gamma}\right] & \text{if } \gamma \neq 2 \\
    \\
    \ln \left(\dfrac{s}{1 + s}\right) & \text{if } \gamma = 2
    \end{cases}
\end{equation}

\section{Axisymmetric Profiles}\label{sec:axisymmetry}

Axisymmetric profiles offer a better approximation for galaxies that exhibit more complex shapes, such as spiral galaxies. In this case, the mass distribution depends on both the radius $R$ in the equatorial plane and the vertical coordinate $z$ perpendicular to this plane. The Laplacian in the Poisson equation then takes the following form:

\begin{equation}
\label{eq:poisson_axisymmetrique}
\nabla^2 \Phi(R, z) = \frac{1}{R}\frac{\partial}{\partial R}\left(R\frac{ \partial \Phi}{\partial R}\right) + \frac{\partial^2\Phi}{\partial z^2}
\end{equation}

Certain axisymmetric profiles admit analytical solutions, such as the Kuzmin profile or the Miyamoto-Nagai profile~\cite{miyamoto1975three}. More generally, solving the Poisson equation for this class of profiles requires numerical methods. In particular, we will be interested in Section~\ref{sec:disk} in the model of the thick exponential disk, which does not admit an analytical solution.

\subsection{Exponential Disk}

The mass distribution of the stellar disk of most galaxies is well represented by an exponential radial profile~\cite{freeman1970disks}

\begin{equation*}
\Sigma(R) = \Sigma_0 \exp{\left(-\frac{R}{R_d}\right)}\text{,}
\end{equation*}
where $\Sigma$ is the surface density, $\Sigma_0$ the surface density at the center of the disk, $R$ is the radius within the disk, and $R_d$ is a scalelength of the disk. These disks can have variable vertical distributions, which are commonly modeled with a hyperbolic secant function $\text{sech}^n = \cosh^{-n}$. Thus, to fully describe galaxy disks, the following formula is used~\cite{smith2015simple}:

\begin{equation}
\label{eq:exp_disk_general}
\rho(R, z) = \rho_0 \exp{\left(-R/R_d\right)}\cdot \text{sech}^n\left(-|z|/z_d\right)\text{,}
\end{equation}
where $z_d$ is a  scaleheight and $n$ is typically between  $\sim 1$ and $3$. $\rho_0$ is the central mass density of the disk, and is usually normalized by the scalelengths of the model. We note the special case $n \rightarrow \infty$ which describes the doubly exponential disk; the vertical component also decreases exponentially.

\subsection{Thick Exponential Disk}

In our study, we are interested in the case of the thick exponential disk. Particularly we study the particular case when $n=2$. Equation~\eqref{eq:exp_disk_general} thus becomes:

\begin{equation}
    \label{eq:exp-disk}
    \rho(R, z) = \rho_0 \exp{\left(-R/R_d\right)}\cdot \cosh^{-2}{\left(-|z|/z_d\right)}
\end{equation}

Firstly, we want to find the associated dimensionless Poisson equation, and then we will look at the approximate solution of this equation~\eqref{eq:exp-disk-potential}. Using the Laplacian from equation~\eqref{eq:poisson_axisymmetrique}, we get directly:

\begin{equation}
\label{eq:exp-disc-poisson1}
\dfrac{1}{R} \dfrac{\partial}{\partial R} \left( R \dfrac{\partial \Phi}{\partial R}\right) + \dfrac{\partial^2 \Phi}{\partial z^2} = 4\pi G \rho_0 \exp{\left(-R/R_d\right)}\cdot \cosh^{-2}{\left(-|z|/z_d\right)}
\end{equation}
Now, by setting $z' = z/z_d$ and $R' = R/R_d$, we can rewrite~\eqref{eq:exp-disc-poisson1} in the form:

\begin{equation*}
\dfrac{1}{R_{d}^{2}} \dfrac{1}{R'} \dfrac{\partial}{\partial R'} \left(R' \dfrac{\partial \Phi}{\partial R'}\right) + \dfrac{1}{z_{d}^{2}}\dfrac{\partial^2 \Phi}{\partial z'^2} = 4\pi G \rho_0 \exp{-R'} \cosh^{-2}{z'}
\end{equation*}
Finally, if we set $\eta = z_d/R_d $ and $\phi'= \frac{\phi}{G M_d/z_d}$, we get a dimensionless Poisson equation for the thick exponential disk:

\begin{equation}
\label{eq:poisson-exp-disc-final}
\dfrac{1}{R'} \dfrac{\partial}{\partial R'} \left(R' \dfrac{\partial \Phi'}{\partial R'}\right) + \dfrac{1}{\eta^{2}}\dfrac{\partial^2 \Phi'}{\partial z'^2} = e^{-R'} \cosh^{-2}{z'}
\end{equation}

In order to implement a numerical solution for this equation, we follow the results of Appendix A of~\cite{bonetti2021dynamical}. The potential is given by the following equation:

\begin{equation}
\label{eq:exp-disk-potential}
\Phi(R, z) = - 2\pi G \alpha \rho_0 \int_{0}^{\infty} dk J_0 (kR) \dfrac{I_z (k)}{(\alpha^2 + k^2)^{\frac{3}{2}}}\text{,}
\end{equation}
with
\begin{equation*}
I_z(k) = \dfrac{4}{\beta} \left\{ 1 - \dfrac{k}{k+\beta} \left[ e^{-z\beta} {}_2F_1\left(1, 1 + \frac{k}{\beta}; 2 + \frac{k}{\beta}; -e^{-z\beta} \right) + e^{z\beta} {}_2F_1\left(1, 1 + \frac{k}{\beta}; 2 + \frac{k}{\beta}; -e^{z\beta} \right)\right] \right\}
\end{equation*}
where $_2F_1$ is the Gaussian hypergeometric function, defined such that :

\begin{equation}
    \label{eq:hypergeometric}
    _2F_1(a, b, c; z) = \sum_{n=0}^{\infty} \frac{(a)_n (b)_n}{(c)_n}\cdot \dfrac{z^n}{n!}
\end{equation}
where $(a)_n$ is the Pochhammer symbol and $|z| < 1$. Note that we also set $\alpha=1/R_d$ and $\beta = 2/z_d$.



\section{Neural Networks}\label{sec:neural-networks}

\begin{frame}{Perceptron}
    \begin{columns}
        \column{\moit} \textbf{Perceptron}: Model inspired by the functioning of neurons in the human brain, provide a mathematical model that enabled machines to learn from data and make predictions~\cite{rosenblatt1958perceptron}.
        \column{\moit} \begin{figure}[ht]
                                \centering
                                \includegraphics[width=\textwidth]{imgs/Single-Perceptron.png}
                                \caption{Illustration of an LTU, which are the building blocks of neural networks}
                                \label{fig:perceptron}
                            \end{figure}
    \end{columns}
\end{frame}


\begin{frame}{Neural networks}
    \begin{columns}
        \column{\moit} Also known as \emph{multi-layer} perceptron. 
                    \begin{enumerate}
                        \item Input layer for the data $\Vec{x}$
                        \item Hidden layers
                        \item Output layer for the network's prediction $\hat{f}_{\Vec{x}}$
                    \end{enumerate}{}
                     Output of layer $l$ can be expressed as:
                        \begin{equation*}
                            \label{eq:output-any-layer}
                            \Vec{a}^{(l)} = \sigma^{(l)}\left(\mathbf{W}^{(l)} \cdot \Vec{a}^{(l-1)} + \Vec{b}^{(l)}\right)
                        \end{equation*}
        \column{\moit} \begin{figure}[ht]
                            \centering
                            \includegraphics[width=\textwidth]{imgs/Architecture-perceptron-multi-couches-2.png}
                            \caption{Structure of a neural network with three hidden layers.}
                            \label{fig:mlp}
                        \end{figure}
                       
    \end{columns}
\end{frame}

\begin{frame}{Neural Networks}
    A neural network is a mathematical function. Can be described as a series of nested non-linear functions:

\begin{align}
f &\text{: } \mathbb{R}^n \to \mathbb{R}^m\nonumber\\
f &= g \circ f_L \circ f_{L-1} \dots f_2 \circ f_1 (x) \text{ with, }
\end{align}

\begin{align}
f_l &\text{: } \mathbb{R}^{n_l} \to \mathbb{R}^{n_{l-1}}\nonumber\\
f_l&(x) = \sigma^{(l)}(\mathbf{W}^{(l)} \cdot \Vec{x} + b^{(l)})
\end{align}

$n$ : dimension of the input $\Vec{x}$, $m$ : dimension of the output $\Vec{y}$, $g$: \emph{output function}
\end{frame}


\begin{frame}{Loss Function}
    Quantification of the error produced by the network via a \emph{loss function}. Commonly used Mean Squared Error (MSE):
    \begin{equation*}
        \label{eq:def-loss-function}
        \mathcal{L}(\theta) = \frac{1}{2n} \sum_{i=1}^n \left[\hat{y}_{\theta}^{(i)} - y^{(i)}\right]^2
    \end{equation*}
    $\theta$: set of parameters (weights and biases), $\hat{y} = \hat{f}(\Vec{x})$: value predicted, $y=f(\Vec{x})$: \emph{real} value.
\end{frame}


\begin{frame}{Gradient Descent}
    Method to udpdate network's parameters: \emph{Gradient descent}~\cite{cauchymethode}
    \begin{columns}
        \column{\moit} Parameters $\theta$  at step $n+1$ are updated:
            \begin{equation*}
                    \label{eq:gradient-descent}
                    \Vec{\theta}_{n+1} = \Vec{\theta}_n - \eta \nabla \mathcal{L}(\Vec{\theta}_n)
            \end{equation*}
            $\eta \in \mathbb{R}^+$ is called \emph{learning rate}.
        \column{\moit} 
        \begin{figure}
            \centering
            \includegraphics[width=\textwidth]{imgs/desc-grad.png}
            \caption{Influence of the learning rate on the convergence of the loss.}
        \end{figure}
    \end{columns}
\end{frame}


%\begin{frame}{Back-propagation}



%\only<1>{Algorithm that enables the computation of the gradient of the error, thereby permitting the adjustment of network parameters according to~\eqref{eq:gradient-descent}.
%\begin{figure}
%    \centering
%    \begin{tikzpicture}

%        \node[draw,rectangle,minimum height=2cm,minimum width=3cm,align=center] (layer) at (0,0) {Layer $l$};
        
%         \draw[<-] ([yshift=0.5cm]layer.west) -- ++(-3,0) node[midway,above] {$X^{(l)}=a^{(l-1)}$};
%         \draw[->] ([yshift=-0.5cm]layer.west) -- ++(-3,0) node[midway,below] {$\frac{\partial E}{\partial X^{(l)}}$};
        
%         \draw[->] ([yshift=0.5cm]layer.east) -- ++(3,0) node[midway,above] {$a^{(l)}=X^{(l+1)}$};
%         \draw[<-] ([yshift=-0.5cm]layer.east) -- ++(3,0) node[midway,below] {$\frac{\partial E}{\partial a^{(l)}} $};
%     \end{tikzpicture}
%     \caption{Hidden layer $l$ of a neural network.}
%     \label{fig:illustration-layer-l}
% \end{figure}}

% \only<2>{The  algorithm can be summarized in three main steps:
% \begin{enumerate}
%     \item Compute the forward pass for each input-output pair by proceeding from layer 1, the input layer, to layer $L$, the output layer.
%     \item Compute the backpward phase for each input-output pair by proceeding from layer $L$, the output layer, to layer 1, the input layer. 
%     %\begin{enumerate}
%     %   \item Evaluate the error term for the final layer.
%     %   \item Backpropagate the error terms for the hidden layers, starting from the last hidden layer $l=L-1$.
%     %   \item Evaluate the partial derivatives of the error with respect to $w_{ik}^{(l)}$ 
%     %\end{enumerate}
%     \item Update the parameters according to the gadient descent algorithm.
% \end{enumerate}

% One full cycle of this algorithm is called an \emph{epoch}.}
% \end{frame}

\begin{frame}{Important Properties}

    \begin{itemize}
        \item \textbf{Universal Approximation Theorem}: asserts that a neural network with a single hidden layer can approximate any continuous function on compact subsets of $\mathbb{R}^n$, provided that the hidden layer's activation function is non-constant, bounded, and continuous.
        \item \textbf{Automatic Differentiation}: Efficient way of computing derivatives with no approximation error.
    \end{itemize}
    
\end{frame}

\section{Physics-Informed Neural Networks}\label{sec:pinns}


\begin{frame}{Physics-Informed Neural Network (PINN)}
\begin{itemize}
    \item Specificity of PINNs: the loss function takes the form of a physical equation
    \item The PINN method is a meshless solution technique, finding the solutions by minimizing a loss function
    \item Ability to solve problems with very little, or noisy data.
    
\end{itemize}
    
\end{frame}



\begin{frame}{Definition of a PINN}
    \begin{enumerate}
        \item Solution $u(z)$ is approximated by a neural network parameterized by a set of parameters $\theta$
        \item For a general differential equation 
        \begin{equation*}
            \label{eq:pde-exemple}
            u_t + \mathcal{F}[u;\lambda] = 0\text{, } x\in \Omega\text{, } t \in [0, T]\text{,}
        \end{equation*} we set
        \begin{equation*}
            p \coloneqq u_t - \mathcal{F}[u;\lambda] \text{.}
        \end{equation*} This function $p$ is referred to as a \emph{physics-informed neural network}.
        \item The network learns to approximate the solution by finding the set of parameters $\theta$ that minimizes a loss function $\mathcal{L}(\theta)$
    \end{enumerate}
\end{frame}

\begin{frame}{Specific Loss Function}
    In the case of a PINN, the loss function is a sum of three components:

    \begin{itemize}
        \item The PDE residual loss 
        \item The boundary loss
        \item The data loss
    \end{itemize}

    \begin{equation*}
        \label{eq:loss-galaxy}
        \mathcal{L}(\theta) = \dfrac{1}{N_c}\sum^{N_c}_{i=1} \left\|\nabla^2 \hat{\Phi}(z_i) - 4 \pi G \rho(z_i) \right\|^2 + \dfrac{1}{N_{d}}\sum^{N_{d}}_{i=1} \left\|\hat{\Phi}(z_i) - \Phi_i \right\|^2
    \end{equation*}


\end{frame}

% \begin{frame}{Applications}
%     \begin{itemize}
%         \item Forward Problem
%         \item Inverse Problem
%     \end{itemize}
% \end{frame}
\begin{frame}{Summary}
    \begin{figure}[ht]
    \centering
    \includegraphics[scale=0.2]{imgs/training-pinn-schema.png}
    \caption{Structure and training procedure of a PINN. \textit{Illustration from~\cite{cuomo_scientific_2022}}.}
    \label{fig:loss-pinn}
\end{figure}

\end{frame}

\section{Applications}\label{ch:applications}


\subsection{Hernquist Model}\label{sec:hernquist}

\begin{frame}{Hernquist Model}
        The differential equation  we aim to solve:

    \begin{equation*}
        \dfrac{\dd}{\dd s}\left(s^2 \dfrac{\dd \Phi'}{\dd s}\right) = \dfrac{2s}{(s+1)^3}
    \end{equation*} which yields the following loss function for the PINN:
    \begin{equation*}
            \mathcal{L}(\theta) = \dfrac{1}{N_c}\sum^{N_c}_{i=1} \left|\dfrac{\dd}{\dd s_i}\left(s_{i}^{2} \dfrac{\dd \hat{\Phi}'}{\dd s_i}\right) - \dfrac{2s_i}{(s_i+1)^3} \right|^2 + \dfrac{1}{N_d}\sum^{N_d}_{i=1} \left|\hat{\Phi}'(s_i) - \Phi'_i \right|^2
    \end{equation*}
\end{frame}

\begin{frame}{Hernquist - Training}
\begin{columns}
    \column{\moit}
    \begin{figure}
        \centering
        \includegraphics[width=\textwidth]{imgs/training-points-hernquist.png}
        \caption{Configuration of training points in the spatial domain $s \in [0, 1000]$.}
        \label{fig:training-points-hernquist}
    \end{figure}
    \column{\moit}
    \begin{figure}
        \centering
        \includegraphics[width=\textwidth]{imgs/error-hernquist.png}
        \caption{Evolution of training and validation error functions.}
        \label{fig:losses-hernquist}
    \end{figure}
\end{columns}
\end{frame}

\begin{frame}{Hernquist - Results}

\begin{figure}
    \centering
        \begin{subfigure}[b]{0.49\textwidth}
        \centering
        \includegraphics[width=\textwidth]{imgs/test-plot-hernquist.png}
        \caption{Comparison between the actual value of the potential and that predicted by the PINN on a test domain $s \in [0, 100]$.}
        \label{fig:test-plot-hernquist}
        \end{subfigure}
    \hfill
    \begin{subfigure}[b]{0.49\textwidth}
        \centering
        \includegraphics[width=\textwidth]{imgs/relative-error-hernquist.png}
        \caption{Relative error along the domain $s$. The average error over the entire domain is $1.71$\%.}
        \label{fig:relative-error-hernquist}
    \end{subfigure}
    \caption{As illustrated by the two figures presented, a PINN is capable of predicting with acceptable accuracy the value of the Hernquist potential $\Phi$ at a given point in a domain in which it has been trained.}
    \label{fig:three graphs}
\end{figure}
\end{frame}


\begin{frame}{Dehnen Model}
\only<1>{The differential equation we want to solve:

\begin{equation*}
    \dfrac{\dd}{\dd s}\left(s^2 \dfrac{\dd \Phi'}{\dd s}\right) = \dfrac{2s^{2-\gamma}}{(1+s)^{4-\gamma}}
\end{equation*}}

The exponent $\gamma$ drastically changes the dynamics of the potential:
\only<2>{
\begin{figure}
\centering
\includegraphics[width=0.6\textwidth]{imgs/gamma-vs-x0-dehnen.png}
\caption{Value of the gravitational potential at $s_0 = 0.01$ for different values of $\gamma \in [0, 3[$. Predicting the potential value at $s_0=0$ is a complex task for the PINN given the high sensitivity of $\Phi(s_0, \gamma)$ to the value of $\gamma$.}
\label{fig:gamma-vs-x0-dehnen}
\end{figure}}
\end{frame}
\subsection{Dehnen Model}\label{sec:dehnen}


\begin{frame}{Dehnen - Training}
    \begin{columns}
        \column{\moit} \begin{figure}
            \centering
            \includegraphics[width=\textwidth]{imgs/training-points-dehnen.png}
            \caption{Distribution of training and validation points on the domain.}
            \label{fig:training-points-dehnen}
        \end{figure}
        \column{\moit} \begin{itemize}
            \item Problems to obtain satisfactory results
            \item No observable improvement by manually tuning hyperparameters
            \item Change the penality of loss function to MAE
        \end{itemize}
    \end{columns}
\end{frame}



\begin{frame}{Dehnen - Results}
    \begin{columns}
    \column{\moit}
        \begin{figure}
            \centering
            \includegraphics[width=\textwidth]{imgs/test-plot-dehnen.png}
            \caption{Comparison between the real value of the potential and that predicted by the PINN on a test domain $s \in [0, 10]$ for different values of $\gamma$.}
            \label{fig:test-plot-dehnen}
        \end{figure}
    \column{\moit} 
        \begin{figure}
            \centering
            \includegraphics[width=\textwidth]{imgs/relative-error-dehnen.png}
            \caption{Relative error on the $s \times \gamma$ grid. The relative error can reach nearly 20\%.}
            \label{fig:relative-error-dehnen}
        \end{figure}
    \end{columns}
\end{frame}



\subsection{Thick Exponential Disk}\label{sec:disk}

\begin{frame}{Thick Exponential Disk Model}
    We aim to solve the following differential equation:
    \begin{equation*}
        \dfrac{1}{R'} \dfrac{\partial}{\partial R'} \left(R' \dfrac{\partial \Phi'}{\partial R'}\right) + \dfrac{1}{\eta^{2}}\dfrac{\partial^2 \Phi'}{\partial z'^2} = e^{-R'} \cosh^{-2}{z'}
\end{equation*} 
$z' = z/z_d$, $R' = R/R_d$, $\eta = z_d/R_d $ and $\phi'= \frac{\phi}{G M_d/z_d}$
\end{frame}

\begin{frame}{Thick Exponential Disk - Training}
    \begin{columns}
        \column{0.4\textwidth} 
        \begin{itemize}
            \item Wish to obtain the best solution possible
            \item Need to find the best set of hyperparameters
            \item Wish to understand the impact of certain hyperpameters on the error

        \end{itemize}
        \column{0.6\textwidth}
        Simple grid search for hyperparameters fine-tuning
        \begin{table}[h]
        \centering
        \scalebox{0.8}{
        \begin{tabular}{|l|c|}
        \hline
        \textbf{Parameters} & \textbf{Values} \\
            \hline
            \# Neurons & 32, 64, 128 \\
            \hline
            \# Layers & 1, 2, 3, 4, 5, 6 \\
            \hline
            Learning Rate & 1e-4, 1e-5, 1e-6 \\
            \hline
            Loss Func. & mse \\
            \hline
            Activation & Tanh, Sigmoid, SiLU, LogSigmoid \\
            \hline
        \end{tabular}}
        \label{tab:fine-tuning}
        \end{table}
    \end{columns}
\end{frame}

\begin{frame}{Thick Exponential Disk - Results}
    \only<1>{
    Two kind of solutions:
    \begin{enumerate}
        \item Computing an approximate solution efficiently $\rightarrow 3 \times 32$ NN, ~ 80 seconds of computation and average error of $1.85\%$.
        \item Find a model giving the smallest error possible $\rightarrow$ fine-tuning, $6 \times 128$ NN, ~ 20 minutes of training and average error of $0.36\%$.
    \end{enumerate}}
    \only<2>{
    \begin{figure}
        \centering
        \includegraphics[width=0.6\textwidth]{imgs/relative-error-expdisc.png}
        %\caption{Relative error on the $R' \times z'$ grid. The average error over the entire domain is 0.36\%, and the maximum error is 0.99\%.}
        \label{fig:relative-error-expdisc}
    \end{figure}
    Satisfactory result, but parameter $\eta$ is here fixed $\rightarrow$ Need to extend the PINN !
    }
    
\end{frame}

\begin{frame}{Hyperparameters}
    How do hyper-parameters influence the relative error? 
    \only<1>{The activation function:
    \begin{figure}
        \centering
        \includegraphics[scale=0.5]{imgs/function-error-expdisc.png}
        %\caption{Average relative error for certain activation functions.}
        \label{fig:function-error-expdisc}
    \end{figure}
}
    \only<2>{The learning rate:
    \begin{figure}
        \centering
        \includegraphics[scale=0.5]{imgs/learning-rate-expdisc.png}
        %\caption{Average relative error for different learning rates.}
        \label{fig:learning-rate-expdisc}
    \end{figure}
    }

    \only<3>{The number of neurons per layer:
     \begin{figure}
        \centering
        \includegraphics[scale=0.4]{imgs/neurons-error-expdisc.png}
        %\caption{Influence of the number of neurons per hidden layer on the average relative error.}
        %\label{fig:neurons-error-expdisc}
    \end{figure}
    }
    \only<4>{The number of hidden layers:
    \begin{figure}
        \centering
        \includegraphics[scale=0.4]{imgs/layers-error-expdisc.png}
        %\caption{Influence of the number of hidden layers on the average relative error.}
        %\label{fig:layers-error-expdisc}
        \end{figure}
    }
    \only<5>{Is it solely the total number of neurons, or does the architecture have an impact?
    \begin{figure}
        \centering
        \includegraphics[width=0.65\textwidth]{imgs/tot-neurons-error.png}
        %\caption{Evolution of the average relative error as a function of the total number of neurons in the PINNs.}
        \label{fig:tot-neurons-error}
    \end{figure}
    }
\end{frame}


\chapter{Conclusion}\label{ch:conclusion}

In this trailblazing study, we have taken the initial stride towards the development of novel tools for simulating galaxies. The proficiency of Physics-Informed Neural Networks (PINNs) in solving the gravitational Poisson equation~\eqref{eq:poisson_eq_general} has been substantiated for three distinct density profiles, two of which are spherically symmetric and one that is axisymmetric. The initial two models yield comparatively accurate results - averaging 1.71\% and 3.75\% respectively - without necessitating any fine-tuning. Further research could potentially diminish the error of these PINNs, however, given that these two models admit an analytical solution, we did not undertake to optimize them in this particular study. The Poisson equation that describes the exponential thick disc model, on the other hand, does not admit an analytical solution, and hence the associated PINN can serve a genuine practical utility.

To confirm that the error could be considerably reduced, we conducted a rudimentary grid search, according to the parameters displayed in Table~\ref{tab:fine-tuning}. This grid search allowed us to select a PINN model that delivers an average relative error of 0.36\%, with a maximum of merely 0.99\%. These results are obtained for a fixed value of $\eta$ (see equation~\eqref{eq:poisson-exp-disc-final}), a ratio of the model's scale lengths. Despite the success of the two-parameter PINN, it remains to be demonstrated that the extension of this PINN to three dimensions performs as satisfactorily. This extension is an excellent opportunity to underscore once again the adaptability of PINNs. Utilizing techniques such as the finite difference method, it is not straightforward to modify the code to include an additional dimension. With a PINN, however, one merely needs to adjust the input data or possibly the PDE residual.


\section{Outlooks and Future Works}

As previously mentioned, this work is innovative in many respects, and while we have taken the initial step by demonstrating the capability of PINNs to solve certain gravitational potentials, there is still much to be accomplished. In the domain of galaxy modeling, one could first extend the work performed to other non-analytical models, or attempt to solve the Jeans equations, which are used to describe the velocity field of a galaxy. More generally, within the field of PINNs, there are several uncharted paths to investigate. We discuss a few of these in this section.
\subsection{Hyperparameters}
While the Hernquist~\cite{hernquist_analytical_1990} and Dehnen~\cite{dehnen_family_1993} profiles admit analytical solutions, it would nonetheless be necessary to undertake an extensive fine-tuning of the hyperparameters to ascertain whether the PINN can achieve an arbitrarily high precision or whether it is bounded. Additionally, the search for hyperparameters for the exponential thick disc allowed us to study the influence of these parameters on the PINN error (see Figures~\ref{fig:hyperparametres-vs-error} and~\ref{fig:tot-neurons-error}). This empirical exploration enables us to note that in the case of the exponential thick disc, the tanh function generally performs significantly better than the sigmoid or SiLU functions. We also observe that the network width-the number of neurons per layer-seems to cease improving the precision beyond a certain threshold. It appears that for an equivalent number of neurons, it is preferable to increase the network depth, as illustrated in Figure~\ref{fig:tot-neurons-error}. Ultimately, although literature~\cite{he_physics-informed_2020} suggests that the L-BFGS-B optimizer is the most effective under conditions similar to ours, it was unable to solve the Poisson equation for the Hernquist model. 
\par More broadly, a dedicated and rigorous exploration of hyperparameters would be beneficial and could potentially contribute to a deeper understanding of the operational intricacies of PINNs.

\subsection{Theory and Interpretability}
The empirical findings discussed above illustrate the challenges encountered with PINNs. Indeed, there are very few theoretical results that provide bounds on the error made by the PINN. A notable result is that of~\cite{de_ryck_approximation_2021}, which provides a bound on the PINN error as a function of the total number of neurons in the network. In another more recent paper~\cite{de_ryck_error_2023}, the same authors demonstrate that there exists a neural network approximating the classical solution of the Navier-Stokes equation such that the generalization error and the training error of the PINN are arbitrarily small. Explicit bounds on the number of neurons and network weights are also provided, depending on the error tolerance and the Sobolev norms of the underlying Navier-Stokes equation. However, these results are limited to two-layer networks using the tanh activation function and therefore do not elaborate on the influence of network width or depth in relation to the total number of neurons in the network. Ultimately, new theoretical results would be beneficial and necessary to prevent PINNs from remaining somewhat of a black box that cannot be fully deciphered.

\subsection{Inverse Problem}

While in this study we have demonstrated the proficiency of PINNs as solvers for swiftly simulating intricate systems, thereby allowing them to be used for parameter estimation, we have not yet fully harnessed the power of PINNs. Indeed, as outlined in Chapter~\ref{ch:pinns}, when furnished with simulation data of a system, a PINN possesses the capability to discern the parameters of a PDE. This is the so-called Inverse Problem. Therefore, if the end goal is to utilize the solution from a PINN over a given domain for parameter estimation, it naturally prompts the question as to whether it is feasible to utilize the inverse problem for ascertaining the parameters of a potential-density pair.

\appendix 
\chapter{Solving the Poisson equation}\label{app:poisson}
\chapter{The Universal Approximation Theorem}\label{app:approx-theorem}
\chapter{Latin Hypercube Sampling}\label{app:lhs}

The Latin Hypercube Sampling~\cite{eglajs1977new,mckay1979comparison} is a statistical method that allows the generation of a sample of near-random parameter values from a multidimensional probability distribution. 

This method ensures that each sample is uniquely positioned within a multidimensional space, such that no two samples share the same position along any dimension. This is achieved by carefully placing each new sample based on the positions of those already placed, so that no common coordinates are shared within the multidimensional space.

The algorithm can be summarized as follows:

\begin{enumerate}
    \item Set a number of points $N$ to sample, as well as the dimension $d$ of the space.
    \item Divide each dimension into $N$ equal probable intervals.
    \item Construct the point $x_k = (x_k^1, x_k^2, \dots, x_k^d)$ with $k \in \{0, 1, \dots , N-1\}$
    and where $$x_k^j = \frac{\pi_j(k) - u_k^j}{N}$$ $\pi_j(k)$ is the cell selected on the axis of the $j^{\text{th}}$ dimension for point $k$, and $u_k^j$ is the position of the point within this cell.
    \item Repeat the previous step $N$ times; until all points are selected.
    \item Return a $N \times d$ matrix containing the coordinates of all points sampled from the hypercube.
\end{enumerate}

This algorithm ensures a well-distributed sampling of the space, where each sample provides unique information, thereby reducing the total number of simulations required and ensuring that the entire space is more uniformly explored.

\newpage
\Large
\noindent
\textbf{Declaration of authorship} 
\vspace{0.5cm}
\noindent
\normalsize

I hereby declare that the report submitted is my own unaided work. All direct 
or indirect sources used are acknowledged as references. I am aware that the 
Thesis in digital form can be examined for the use of unauthorized aid and in 
order to determine whether the report as a whole or parts incorporated in it may 
be deemed as plagiarism. For the comparison of my work with existing sources I 
agree that it shall be entered in a database where it shall also remain after 
examination, to enable comparison with future Theses submitted. Further rights 
of reproduction and usage, however, are not granted here. This paper was not 
previously presented to another examination board and has not been published.

\bibliographystyle{apalike}
\bibliography{references}
%\printglossary[type=\acronymtype]
\end{document}
